%%%%%%%%%%%%%%%%%%%%%%%%%%%%%%%%%%%%%%%%%%%%%%%%%%%%%%%%%%%%%%%%%%%%%%%%%
% This file is part of the LaTeX sources of the OMDoc 1.3 specification
% Copyright (c) 2006 Michael Kohlhase
% This work is licensed by the Creative Commons Share-Alike license
% see http://creativecommons.org/licenses/by-sa/2.5/ for details
\svnInfo $Id: validating.tex 8685 2010-08-23 08:55:17Z kohlhase $
\svnKeyword $HeadURL: https://svn.omdoc.org/repos/omdoc/branches/omdoc-1.3/doc/spec/validating.tex $
%%%%%%%%%%%%%%%%%%%%%%%%%%%%%%%%%%%%%%%%%%%%%%%%%%%%%%%%%%%%%%%%%%%%%%%%%

\begin{tchapter}[id=validating]{Validating OMDoc Documents}

In {\mychapref{markup-web}} we have briefly discussed the basics of validating {\xml}
documents by {\twintoo{document type}{definition}s} (DTDs) and {\indextoo{schema}ta}. In
this chapter, we will instantiate this discussion with the particulars of
validating\index{validation} {\omdoc} documents.

Generally, DTDs and schemata are {\twintoo{context-free}{grammar}s} for
trees\footnote{Actually, a recent extension of the {\xml} standard ({\xlink}) also allows
  to express graph structures, but the admissibility of graphs is not covered by DTD or
  current schema formalisms.}, that can be used by a {\twindef{validating}{parser}} to
reject {\xml} documents that do not conform to the constraints expressed in the {\omdoc}
DTD or schemata discussed here.

Note that none of these grammars can enforce all constraints that the {\omdoc}
specification in {\mypartref{specification}} of this {\report} imposes on documents.
Therefore grammar-based validation is only a necessary condition for
{\omdoc}-{\defin{validity}}. Still, {\omdoc} documents should be validated to ensure
proper function of {\omdoc} tools, such as the ones discussed in
{\mychapsref{transform-xsl}{projects}}.  Validation against multiple grammars gives the
best results. With the current state of validation technology, there is no clear
recommendation, which of the validation approaches to prefer for {\omdoc}. DTD validation
is currently best supported by standard {\xml} applications and supports default values
for attributes. This allows the author who writes {\omdoc} documents by hand to elide
implicit attributes and make the marked-up text more readable.  {\xml}- and {\relaxng}
schema validation have the advantage that they are namespace-aware and support more
syntactic constraints.  Neither of these support mnemonic {\xml}
entities\twin{XML}{entity}, such as the ones used for {\unicode} characters in {\pmathml},
so that these have to be encoded as {\unicode} code points. Finally {\relaxng} schemata do
not fully support default values for attributes, so that {\omdoc} documents have to be
normalized\footnote{An {\omdoc} document is called {\defin{normalized}}, iff all required
  attributes are present. Given a {\indextoo{DTD}} or {\xml} schema\twin{XML}{schema} that
  specifies default values, there are standard {\xml} tools for {\xml}-normalization that
  can be pipelined to allow {\relaxng} validation, so this is not a grave restriction.} to
be {\relaxng}-valid.

We will now discuss the particulars of the respective validation formats. As the
{\relaxng} schema is the most expressive and readable for humans we consider it as the
{\atwintoo{normative}{grammar}{formalism}} for {\omdoc}, and have included it in
{\myappchapref{rnc}} for reference.

\begin{tsection}[id=validate-dtd]{Validation with Document Type Definitions}
  
  The {\omdoc} document type definition~\cite{OMDocDTD:URL} can be referenced by the
  {\twintoo{public}{identifier}} {\snippet{"-//OMDoc//DTD OMDoc V1.3//EN"}}
  (see~\mysecref{catalog}). The DTD driver file is \url{omdoc.dtd}, which calls various
  {\indextoo{DTD module}s}.
  
  DTD-validating {\xml} parsers are included in almost all {\xml} processors. The author
  uses the open-source {\rxp}~\cite{Tobin:RXP} and
  {\scsys{xmllint}}~\cite{Veillard:libxml2} as stand-alone tools. If required, one may
  validate {\omdoc} documents using an {\sgml} parser such as {\nsgmls}, rather than a
  validating {\xml} parser. In this case an {\sgml} declaration defining the constraints
  of {\xml} applicable to an {\sgml} parser must be used (see~\cite{Clark:csx97} for
  details).

  To allow DTD-validation, {\omdoc} documents should contain a {\indextoo{document
% There's a space missing between ``type'' and ``declaration'', but I don't know
% how to specify it. --Christoph Lange
      type}{declaration}} of the following form:
\begin{center}
\begin{lstlisting}[language=XML,index={DOCTYPE,omdoc}]
<!DOCTYPE omdoc PUBLIC "-//OMDoc//DTD OMDoc V1.3//EN"
                       "http://omdoc.org/dtd/omdoc.dtd">
\end{lstlisting}
\end{center}
The {\indextoo{URI}} may be changed to that of a local copy of the DTD if required, or it
can be dropped altogether if the processing application has access to an {\xml}
catalog\twin{XML}{catalog} (see~\mysecref{catalog}).  Whether it is useful to include
document type declarations in documents in a production environment depends on the
application. If a document is known to be DTD- or even {\omdoc}-valid, then the validation
overhead a {\snippetin{DOCTYPE}} declaration would incur from a validating
parser\footnote{The {\xml} specification requires a validating parser to perform
  validation if a {\snippetin{DOCTYPE}} declaration is present} may be conserved by
dropping it.


\begin{tsubsection}[id=parameter-entities]{Parametrizing the DTD}

  The {\omdoc} DTD makes heavy use of parameter entities\twin{parameter}{entity},
  so we will briefly discuss them to make the discussion self-contained. Parameter
  entity declarations are declarations of the form
\begin{center}
\begin{lstlisting}[language=XML,index={ENTITY}]
<!ENTITY % assertiontype "theorem|proposition|lemma|%otherassertiontype;">
\end{lstlisting}
\end{center}
in the DTD. This one makes the abbreviation {\snippet{\%assertiontype;}} available for the
string ``\snippet{theorem|proposition|lemma|observation}'' (in the DTD of the document in
{\mylstref{internal}}). Note that parameter entities must be fully defined before they can
be referenced\twin{reference}{parameter entity}, so recursion is not possible. If there
are multiple parameter entity declarations, the first one is relevant for the computation
of the replacement text; all later ones are discarded. The {\twintoo{internal}{subset}} of
{\twintoo{document type}{declaration}} is pre-pended to the external {\indextoo{DTD}}, so
that parameter entity declarations in the {\twintoo{internal}{subset}} overwrite the ones
in the {\twintoo{external}{subset}}.

The (external) DTD specified in the {\snippetin{DOCTYPE}} declaration can be enhanced or
modified by adding declarations in square brackets after the {\indextoo{DTD}}
{\indextoo{URI}}. This part of the DTD is called the
{\twintoo{internal}{subset}}\atwin{internal}{subset}{DTD} of the {\snippetin{DOCTYPE}}
declaration, see {\mylstref{internal}} for an example, which modifies the
{\twintoo{parameter}{entity}} {\snippet{\%other\-assertiontype;}} supplied by the {\omdoc}
DTD to extend the possible values of the {\attribute{type}{assertion}} attribute in the
{\element{assertion}} element for this document. As a consequence, the
{\element{assertion}} element with the non-standard value for the
{\attribute{type}{assertion}} attribute is DTD-valid with the modified internal DTD
subset.

\begin{lstlisting}[label=lst:internal,language=XML,morekeywords={omdoc},mathescape,
  caption={A Document Type Declaration with Internal Subset},
  index={DOCTYPE,ENTITY,ELEMENT,ATTLIST,omdoc,PUBLIC}]
<!DOCTYPE omdoc PUBLIC "-//OMDoc//DTD OMDoc V1.3//EN"
                       "http://omdoc.org/omdoc.dtd" 
  [<!ENTITY % otherassertiontype "observation">]>
$\ldots$
<assertion type="observation">$\ldots$</assertion>
$\ldots$
\end{lstlisting}
\end{tsubsection}

\begin{tsubsection}[id=dtd-normalization]{DTD-based Normalization}
  
  Note that if a {\omdoc} fragment is parsed without a DTD, i.e. as a
  {\indextoo{well-formed}} {\xml} {\indextoo{fragment}}, then the default attribute
  values\twin{default value}{attribute} will not be added to the {\xml}
  {\twintoo{information}{set}}. So simply dropping the {\twintoo{DOCTYPE}{declaration}}
  may change the semantics of the document, and {\omdoc} documents should be
  normalized\footnote{The process of DTD-normalization expands all parsed {\xml} entities,
    and adds all default attribute values}\twin{DTD}{normalization} first.  Normalized
  {\omdoc} documents should carry the {\attribute{standalone}{?xml}} attribute in the
  {\xml} processing instruction, so that a normalized {\omdoc} document has the form given
  in {\mylstref{normalized}}.

\begin{lstlisting}[label=lst:normalized,language=XML,morekeywords={omdoc},mathescape,
  caption={A normalized {\omdoc} document without DTD},index={xml,omdoc}]
<?xml version="1.0" standalone="yes"?>
<omdoc xml:id="something" version="1.3" xmlns="http://omdoc.org/ns">
 $\ldots$
</omdoc>
\end{lstlisting}

The attribute {\attribute{version}{omdoc}} and the namespace declaration
{\snippetin{xmlns}} are fixed by the DTD, and need not be explicitly provided if the
document has a {\snippetin{DOCTYPE}} declaration.
\end{tsubsection}


\begin{tsubsection}[id=modularization]{Modularization}

  In {\omdocv{1.3}} the DTD has been modularized according to the W3C conventions
  for DTD {\indextoo{modularization}}~\cite{AltBou:mox01}. This partitions the DTD
  into  {\twindef{DTD}{module}s} that correspond to the {\omdoc} modules
  discussed in {\mypartref{specification}} of this {\report}. 

  These DTD modules can be deselected from the {\omdoc} DTD by changing the
  {\defemph{module inclusion entities}}\twin{module inclusion}{entity} in the
  local subset of the document type declaration. In the following declaration, the
  module {\PFmodule{DTD}} (see {\mychapref{proofs}}) has been deselected,
  presumably, as the document does not contain proofs.

\begin{lstlisting}[index={omdoc,DOCTYPE,ENTITY,INGNORE}]
<!DOCTYPE omdoc PUBLIC "-//OMDoc//DTD OMDoc V1.3//EN"
                             "http://omdoc.org/dtd/omdoc.dtd"
  [<!ENTITY % omdoc.pf.module "IGNORE">]>
\end{lstlisting}

Module inclusion entities have the form {\snippet{\%omdoc.\llquote{ModId}.module;}}, where
{\llquote{ModId}} stands for the lower-cased module identifier. The {\omdoc} DTD
repository contains DTD driver files for all the sub-languages discussed in
{\mysecref{sub-languages}}, which contain the relevant module inclusion entity
settings. These are contained in the files {\snippet{omdoc-\llquote{SlId}.dtd}}, where
{\snippet{\llquote{SlId}}} stands for the sub-language identifier.

Except for their use in making the {\omdoc} DTD more manageable, DTD modules also allow to
include {\omdoc} functionality into other document types, extending {\omdoc} with new
functionality encapsulated into modules or upgrading selected {\omdoc} modules
individually. To aid this process, we will briefly describe the module structure.
Following~\cite{AltBou:mox01}, DTD modules come in two parts, since we have inter-module
recursion. The problem is for instance that the {\element{omlet}} element can occur in
{\twintoo{mathematical}{text}s} ({\snippet{mtext}}), but also contains {\snippet{mtext}},
which is also needed in other modules. Thus the modules cannot trivially be linearized.
Therefore the DTD driver includes an entity file {\snippet{\llquote{ModId}.ent}} for each
module {\snippet{\llquote{ModId}}}, before it includes the grammar rules in the
{\twindef{element}{module}s} {\snippet{\llquote{ModId}.mod}} themselves. The entity files
set up parameter entities for the qualified names and content models that are needed in
the grammar rules of other modules.
\end{tsubsection}

\begin{tsubsection}[id=namespace-magic]{Namespace Prefixes for {\omdoc} elements}
  
  Document type definitions\twin{document type}{definition} do not natively support {\xml}
  namespaces\twin{XML}{namespace}. However, clever coding tricks allow them to simulate
  namespaces to a certain extent. The {\omdoc} DTD follows the approach
  of~\cite{AltBou:mox01} that parametrizes namespace prefixes in element names to deal
  gracefully with syntactic effects of namespaced documents like we have in {\omdoc}.
  
  Recall that element names are  {\twindef{qualified}{name}s}, i.e. pairs
  consisting of a {\twintoo{namespace}{URI}} and a {\twintoo{local}{name}}. To save typing
  effort, {\xml} allows to abbreviate qualified names by namespace declarations via
  {\snippetin{xmlns}} pseudo-attribute: the element and all its descendants are in this
  namespace, unless they have a namespace attribute of their own or there is a namespace
  declaration in a closer ancestor that overwrites it.  Similarly, a
  {\twintoo{namespace}{abbreviation}} can be declared on any element by an attribute of
  the form {\snippet{xmlns:nsa="nsURI"}}, where {\snippet{nsa}} is a name space
  abbreviation, i.e. a simple name, and {\snippet{nsURI}} is the URI of the namespace.  In
  the scope of this declaration (in all descendants, where it is not overwritten) a
  qualified name {\snippet{nsa:n}} denotes the qualified name {\snippet{nsURI:n}}.

  The mechanisms described in~\cite{AltBou:mox01} provide a way to allow for namespace
  declarations even in the (namespace-agnostic) DTD setting simply by setting a
  {\indextoo{parameter entity}}. If {\snippetin{NS.prefixed}} is declared to be
  {\snippetin{INCLUDE}}, using a declaration such as {\snippet{<!ENTITY \% NS.prefixed
      "INCLUDE">}} either in the local subset of the {\snippetin{DOCTYPE}} declaration, or
  in the DTD file that is including the {\omdoc} DTD, or the DTD modules presented in this
  appendix, then all {\omdoc} elements should be used with a prefix, for example
  {\snippet{<omdoc:definition>}}, {\snippet{<omdoc:CMP>}}, etc. The prefix defaults to
  {\snippetin{omdoc:}} but another prefix may be declared by declaring in addition the
  parameter entity {\snippetin{omdoc.prefix}}. For example, {\snippet{<!ENTITY \%
      omdoc.prefix "o">}} would set the prefix for the {\omdoc} namespace to
  {\snippetin{o:}}.

Note that while the Namespaces Recommendation~\cite{BraHol:xmlns99} provides
mechanisms to change the prefix at arbitrary points in the document, this
flexibility is not provided in this DTD (and is probably not possible to specify
in any DTD).  Thus, if a namespace prefix is being used for {\omdoc} elements, so
that for example the root element is:
\begin{lstlisting}[index={omdoc:omdoc},mathescape]
<omdoc:omdoc xmlns:omdoc="http://omdoc.org/ns">
$\ldots$
</omdoc:omdoc>
\end{lstlisting}
then the prefix must be declared in the local subset of the DTD, as follows:
\begin{lstlisting}[index={omdoc:omdoc,DOCTYPE,ENTITY,NS.prefixed,INCLUDE}]
<!DOCTYPE omdoc:omdoc PUBLIC "-//OMDoc//DTD OMDoc V1.3//EN"
                             "http://omdoc.org/dtd/omdoc.dtd"
  [<!ENTITY % NS.prefixed "INCLUDE"><!ENTITY % omdoc.prefix "omdoc">]>
\end{lstlisting}

The {\omdoc} DTD references six namespaces:
  \begin{center}
    \begin{tabular}{|l|l|l|}\hline
      language         & namespace                                & prefix \\\hline\hline
      {\mathml}        & \url{http://www.w3.org/1998/Math/MathML} & {\snippet{m:}}\\\hline
      {\openmath}      & \url{http://www.openmath.org/OpenMath}   & {\snippet{om:}}\\\hline
      {\xslt}          & \url{http://www.w3.org/1999/XSL/Transform}   & {\snippet{xsl:}}\\\hline
      Dublin Core      & \url{http://purl.org/dc/elements/1.1/}   & {\snippet{dc:}}\\\hline
      Creative Commons & \url{http://creativecommons.org/ns}      & {\snippet{cc:}}\\\hline
      {\omdoc}         & \url{http://omdoc.org/ns}       & {\snippet{omdoc:}}\\\hline
    \end{tabular}
  \end{center}
These prefixes can be changed just like the {\omdoc} prefix above. 
\end{tsubsection}
\end{tsection}

\begin{tsection}[id=validate-rnc]{Validation with RelaxNG Schemata}

  {\relaxng}~\cite{Vlist:rng03} is a relatively young approach to validation developed
  outside the {\indextoo{W3C}}, whose {\xml} schema paradigm was deemed overburdened. As a
  consequence, {\relaxng} only concerns itself with validation, and leaves out typing,
  normalization, and entities. {\relaxng} schemata come in two forms, in {\xml} syntax
  (file name extension {\snippetin{.rng}}) and in compact syntax (file name extension
  {\snippetin{.rnc}}). We provide the {\relaxng} schema~\cite{OMDocRNC:URL} as the
  normative validation schema for {\omdoc}. As compact syntax is more readily
  understandable by humans, we have reprinted it as the normative grammar for {\omdoc}
  documents in {\myappchapref{rnc}}. Just as in the case for the {\omdoc} DTD, we provide
  schemata for the {\omdoc} sub-languages discussed in {\mysecref{sub-languages}}. These are
  contained in the driver files {\snippet{omdoc-\llquote{SlId}.rnc}}, where
  {\snippet{\llquote{SlId}}} stands for the sub-language identifier.
  
  There is currently no standard way to associate a {\relaxng} schema with an {\xml}
  document\footnote{In fact this is not an omission, but a conscious design decision on
    the part of the {\relaxng} developers.}; thus validation tools (see
  \url{http://relaxng.org} for an overview) have to be given a grammar as an explicit
  argument. One consequence of this is that the information that an {\omdoc} document is
  intended for an {\omdoc} sub-languages must be managed outside the document separately
  from the document.

  There are various validators for {\relaxng} schemata, the author uses the open-source
  {\scsys{xmllint}}~\cite{Veillard:libxml2} as a stand-alone tool, and the
  {\twintoo{nXML}{mode}}~\cite{nxml-mode:URL} for the {\emacs} editor~\cite{Stallman:em02}
  for editing {\xml} files, as it provides powerful {\relaxng}-based editing support
  (validation, completion, etc.).
\end{tsection}

\begin{tsection}[id=xsd:validation]{Validation with XML Schema}
  
  For validation\footnote{There are many schema-validating {\xml}
    parsers\atwin{validating}{XML}{parser}, the author uses the open-source
    {\scsys{xmllint}}~\cite{Veillard:libxml2}.} with respect to {\xml}
  {\indextoo{schema}ta} (see~\cite{XML:Schema}) we provide an {\xml} schema for
  {\omdoc}~\cite{OMDocXSD:URL}, which is generated from the {\relaxng} schema in
  {\myappchapref{rnc}} via the {\scsys{trang}} system described above. Again,
  {\indextoo{schema}ta} for the sub-languages discussed in {\mysecref{sub-languages}} are
  provided as {\snippet{omdoc-\llquote{SlId}.rnc}}, where {\snippet{\llquote{SlId}}}
  stands for the sub-language identifier.

  To associate an {\xml} schema with an element, we need to decorate it with an
  {\attribute[ns-attr=xsi]{schemaLocation}{omdoc}} attribute and the namespace declaration
  for {\xml} {\twintoo{schema}{instance}s}. In {\mylstref{xml-schema}} we have done this
  for the top-level {\element{omdoc}} element, and thus for the whole document. Note that
  this mechanism makes mixing {\xml} vocabularies much simpler than with DTDs, that can
  only be associated with whole documents.

\begin{lstlisting}[label=lst:xml-schema,mathescape,
  caption={An \protect{\xml} document with an {\xml} Schema.},
  index={omdoc,xmlns,xsi,schemaLocation}]
<?xml version="1.0"?>
<omdoc xml:id="example.with.schema"
       xmlns:xsi="http://www.w3.org/2001/XMLSchema-instance" 
       xsi:schemaLocation="http://omdoc.org/ns
                           http://omdoc.org/xsd/omdoc.xsd">
 $\ldots$
</omdoc>
\end{lstlisting}
\end{tsection}

\end{tchapter}
%%% Local Variables: 
%%% mode: latex
%%% TeX-master: "omdoc"
%%% End: 

% LocalWords:  DTD dtd omdoc nsgmls DOCTYPE standalone xml xmlns lst ATTLIST NS
% LocalWords:  morekeywords otherassertiontype omlet mtext ent OM dtd dtd ns
% LocalWords:  reorg assertiontype Namespaces DC om xsd XSDs xsi schemaLocation
% LocalWords:  appinfo emph xs mdiff emphstyle complexType omtextType CMP FMP
% LocalWords:  maxOccurs minOccurs dag xref OMOBJ attributeGroup attrib anyURI
% LocalWords:  rnc xsl xmllint deselected INGNORE xxx dc cc rng nsa nsURI ns ta
% LocalWords:  mathescape attr ModId SlId nXML trang dtd dtd ns dtd dtd ns dtd
% LocalWords:  dtd ns dtd dtd ns dtd dtd dtd dtd ns
