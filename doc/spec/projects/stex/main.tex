%%%%%%%%%%%%%%%%%%%%%%%%%%%%%%%%%%%%%%%%%%%%%%%%%%%%%%%%%%%%%%%%%%%%%%%%%
% This file is part of the LaTeX sources of the OMDoc 1.3 specifiation
% Copyright (c) 2006 Michael Kohlhase
% This work is licensed by the Creative Commons Share-Alike license
% see http://creativecommons.org/licenses/by-sa/2.5/ for details
\svnInfo $Id: main.tex 8453 2009-08-04 09:58:26Z kohlhase $
\svnKeyword $HeadURL: https://svn.omdoc.org/repos/omdoc/branches/omdoc-1.3/doc/spec/projects/stex/main.tex $
%%%%%%%%%%%%%%%%%%%%%%%%%%%%%%%%%%%%%%%%%%%%%%%%%%%%%%%%%%%%%%%%%%%%%%%%%

\section{\protect\stex: A {\LaTeX}-Based Workflow for OMDoc}
\begin{erratum}[reported-by=Christoph Lange,date=2006-12-12]{The domain is \texttt{kwarc.eecs.iu-bremen.de}}
\begin{project}{stex}{http://kwarc.iu-bremen.de/projects/stex/}
\pauthors{Michael Kohlhase}
\pinstitute{Computer Science, International University Bremen}
\end{project}
\end{erratum}

One of the reasons why {\omdoc} has not been widely employed for representing mathematics
on the web and in scientific publications, may be that the technical communities that need
high-quality methods for publishing mathematics already have an established method which
yields excellent results --- the {\TeX/\LaTeX} system. A large part of mathematical
knowledge is prepared in the form of {\TeX}/{\LaTeX} documents.

We present {\stex} (Semantic {\TeX}) a collection of macro packages for {\TeX/\LaTeX}
together with a transformation engine that transforms {\stex} documents to the {\omdoc}
format. {\stex} extends the familiar and time-tried {\LaTeX} workflow until the last step
of Internet publication of the material: documents can be authored and maintained in
{\stex} using a simple text editor, a process most technical authors are well familiar
with. Only the last (publishing) step (which is fully automatic) transforms the document
into the unfamiliar {\xml} world. Thus, {\stex} can serve as a conceptual interface
between the document author and {\omdoc}-based systems: Technically, {\stex} documents are
transformed into {\omdoc}, but conceptually, the ability to semantically annotate the
source document is sufficient.

\begin{tsubsection}{Recap of the {\TeX/\LaTeX} System}
{\TeX}~\cite{Knuth:ttb84} is a document presentation format that combines complex
page-description primitives with a powerful macro-expansion facility, which is utilized in
{\LaTeX} (essentially a set of {\TeX} macro packages, see~\cite{Lamport:ladps94}) to
achieve more content-oriented markup that can be adapted to particular tastes via
specialized document styles. It is safe to say that {\LaTeX} largely restricts content
markup to the document structure\footnote{supplying macros e.g. for sections, paragraphs,
  theorems, definitions, etc.}, and graphics, leaving the user with the presentational
{\TeX} primitives for mathematical formulae. Therefore, even though {\LaTeX} goes a great
step into the direction of a content/context markup format, it lacks infrastructure for
marking up the functional structure of formulae and mathematical statements, and their
dependence on and contribution to the mathematical context.
 
But the adaptable syntax of {\TeX/\LaTeX} and their tightly integrated programming
features have distinct advantages on the authoring side:
\begin{itemize}
\item The {\TeX/\LaTeX} syntax is much more compact than {\omdoc}, and if needed, the
  community develops {\LaTeX} packages that supply new functionality with a succinct and
  intuitive syntax.
\item The user can define ad-hoc abbreviations and bind them to new control sequences to
  structure the source code.
\item The {\TeX/\LaTeX} community has a vast collection of language extensions and best
  practice examples for every conceivable publication purpose. Additionally, there is an
  established and very active developer community that maintains these.
\item A host of software systems are centered around the {\TeX/\LaTeX} language that make
  authoring content easier: many editors have special modes for {\LaTeX}, there are
  spelling/style/grammar checkers, transformers to other markup formats, etc.
\end{itemize}
 
In other words, the technical community is heavily invested in the whole
{\indextoo{workflow}}, and technical know-how about the format permeates the
community. Since all of this would need to be re-established for an {\omdoc}-based
workflow, the community is slow to take up {\omdoc} over {\TeX/\LaTeX}, even in light of
the advantages detailed in this book.
\end{tsubsection} 

\begin{tsubsection}{A {\LaTeX}-based Workflow for {\xml}-based Mathematical Documents}
 
  An elegant way of sidestepping most of the problems inherent in transitioning from a
  {\LaTeX}-based to an {\xml}-based workflow is to combine both and take advantage of the
  respective values.
 
  The key ingredient in this approach is a system that can transform {\TeX/\LaTeX}
  documents to their corresponding {\xml}-based counterparts. That way, {\xml}-documents
  can be authored and prototyped in the {\LaTeX} workflow, and transformed to {\xml} for
  publication and added-value services.
 
  There are various attempts to solve the {\TeX/\LaTeX} to {\xml} transformation problem;
  the most mature is probably Bruce Miller's {\latexml} system~\cite{Miller:latexml}. It
  consists of two parts: a re-implementation of the {\TeX} {\indextoo{analyzer}} with all
  of its intricacies, and an extensible {\xml} {\indextoo{emitter}} (the component that
  assembles the output of the parser). Since {\LaTeX} style files are (ultimately)
  programmed in {\TeX}, the {\TeX} analyzer can handle all {\TeX}
  extensions\footnote{i.e. all macros, environments, and syntax extensions used int the
    source document}, including all of {\LaTeX}. Thus the {\latexml} parser can handle all
  of {\TeX/\LaTeX}, if the emitter is extensible, which is guaranteed by the {\latexml}
  binding language: To transform a {\TeX/\LaTeX} document to a given {\xml} format, all
  {\TeX} extensions must have ``{\latexml} bindings''\twin{LaTeXML}{binding},
  i.e. directives to the {\latexml} emitter that specify the target representation in
  {\xml}.

  The {\stex} system that we present here supplies a set of {\TeX/\LaTeX} packages and the
  respective {\latexml} bindings that allow to add enough structural information in the
  {\TeX/\LaTeX} sources, so that the {\latexml} system can transform them into documents
  in {\omdoc} format.
\end{tsubsection}

\begin{tsubsection}{Content Markup of Mathematical Formulae in {\TeX/\LaTeX}}

  The main problem here is that run-of-the-mill {\TeX/\LaTeX} only specifies the
  presentation (i.e. what formulae look like) and not their content (their functional
  structure). Unfortunately, there are no universal methods (yet) to infer the latter from
  the former.  Consider for instance the following ``standard notations''\footnote{The
    first one is standard e.g. in Germany and the US, the third one in France, and the
    last one in Russia} for binomial coefficients: $\left(n\atop k\right)$, $_nC^k$,
  ${\cal C}^n_k$, and ${\cal C}^k_n$ all mean the same thing: $n!\over k!(n-k)!$. This
  shows that we cannot hope to reliably recover the functional structure (in our case the
  fact that the expression is constructed by applying the binomial function to the
  arguments $n$ and $k$) from the presentation alone short of understanding the underlying
  mathematics.
 
  The apparent solution to this problem is to dump the extra work on the author (after all
  she knows what she is talking about) and give her the chance to specify the intended
  structure. The markup infrastructure supplied by the {\stex} collection lets the author
  do this without changing the visual appearance, so that the {\LaTeX} workflow is not
  disrupted. We speak of {\twindef{semantic}{preloading}} for this process. For instance,
  we can now write
\begin{equation}\label{cmathml-sum}\footnotesize
  \verb|\CSum{k}1\infty{\Cexp{x}k}| \quad\hbox{\normalsize instead of}\quad
  \verb|\sum_{k=1}^\infty x^k|
\end{equation}
for the mathematical expression $\sum_{k=1}^\infty x^k$. In the first form, we specify
that we are applying a function (\verb|CSumLimits| $\hat=$ sum with limits) to four
arguments: ({\sl{i}}) the bound variable $k$ ({\sl{ii}}) the number 1 ({\sl{iii}})
$\infty$ ({\sl{iv}}) \verb|\Cexp{x}k| (i.e. $x$ to the power $k$).  In the second form, we
merely specify hat {\LaTeX} should draw a capital sigma character ($\Sigma$) whose
subscript is the equation $k=1$ and whose superscript is $\infty$. Then it should place
next to it an $x$ with an upper index $k$.

Of course human readers (who understand the math) can infer the content structure from the
expression $\sum_{k=1}^\infty x^k$ of the right-hand representation in
(\ref{cmathml-sum}), but a computer program (who does not understand the math or know the
context in which it was encountered) cannot. However, a converter like {\latexml} can
infer this from the left-hand {\LaTeX} structure with the help of the curly braces that
indicate the argument structure. This technique is nothing new in the {\TeX/\LaTeX} world,
we use the term ``{\twindef{semantic}{macro}}'' for a macro whose expansion stands for a
mathematical object. The {\stex} collection provides semantic macros for all {\cmathml}
elements together with {\latexml} bindings that allow to convert {\stex} formulae into
{\mathml}.
\end{tsubsection}

\begin{tsubsection}{Theories and Inheritance of Semantic Macros}
  Semantic macros are traditionally used to make {\TeX/\LaTeX} code more
  portable. However, the {\TeX/\LaTeX} scoping model (macro definitions are scoped either
  in the local group or to the end of the document), does not mirror mathematical
  practice, where notations are scoped by mathematical environments like statements,
  theories, or such (see~\cite{Kohlhase:smtl05} for a discussion and examples). Therefore
  the {\stex} collection provides an infrastructure to define, scope, and inherit semantic
  macros.

  In a nutshell, the {\stex} {\verb|symdef|} macro is a variant of the usual
  {\verb|newcommand|}, only that it is scoped differently: The visibility of the defined
  macros is explicitly specified by the {\snippet{module}} environment that corresponds to
  the {\omdoc} {\element{theory}} element.  For this the {\snippet{module}} environment
  takes the optional {\snippet{KeyVal}} arguments {\snippet{id}} for specifying the theory
  name and {\snippet{uses}} for the {\atwintoo{semantic}{inheritance}{relation}}. For
  instance a {\snippet{module}} that begins with
\begin{lstlisting}
  \begin{module}[id=foo,uses={bar,baz}]
\end{lstlisting}
restricts the scope of the semantic macros defined by the {\verb|\symdef|} form to the end of
this module given by the corresponding {\verb|\end{module}|}, and to any other
{\snippet{module}} environment that has {\snippet{[uses=\{\ldots,foo,\ldots\}]}} in its
declaration. In our example the semantic macros from the modules {\snippet{bar}}
and {\snippet{baz}} are inherited as well as the ones that are inherited by these modules.

We will use a simple module for natural number arithmetics as an example. It
declares a new semantic macro for summation while drawing on the basic
operations like $+$ and $-$ from {\LaTeX}.  {\verb|\Sumfromto|} allows us to
express an expression like \erratumReplace[reported-by=Christoph
Lange,date=2006-12-12]{correct example given}{$\sum_{i=1}^nx^i$}{$\sum_{i=1}^n
  2i-1$} as {\verb|\Sumfromto{i}1n{2i-1}|}. In this example we have also made
use of a local semantic symbol for $n$, which is treated as an arbitrary (but
fixed) symbol (compare with the use of {\verb|\arbitraryn|} below, which is a
new --- semantically different --- symbol).
\begin{lstlisting}
 \begin{module}[id=arith]
   \symdef{Sumfromto}[4]{\sum_{#1=#2}^{#3}{#4}}
   \symdef[local]{arbitraryn}{n}
   What is the sum of the first $\arbitraryn$ odd numbers, i.e.
   $\Sumfromto{i}1\arbitraryn{2i-1}?$
 \end{module}
\end{lstlisting}
is formatted by {\sTeX} to
\begin{quote}\hrule
  What is the sum of the first $n$ odd numbers, i.e.  $\sum_{i=1}^n{2i-1}$?\hrule
\end{quote}
Moreover, the semantic macro {\verb|Sumfromto|} can be used in all {\snippet{module}}
environments that import it via its {\snippet{uses}} keyword. Thus {\stex} provides
sufficient functionality to mark up {\omdoc} theories with their scoping rules in a very
direct and natural manner. The rest of the {\omdoc} elements can be modeled by {\LaTeX}
environments and macros in a straightforward manner.

The {\stex} macro packages have been validated together with a case
study~\cite{Kohlhase:smtl05}, where we semantically preloaded the course materials for a
two-semester course ``General Computer Science I\&II'' at International University Bremen
and transform them to the {\omdoc}, so that they can be used in the {\activemath} system
(see {\mysecref{activemath}}).
\end{tsubsection}
%%% Local Variables: 
%%% mode: latex
%%% TeX-master: "../../omdoc"
%%% End: 

% LocalWords:  stex LaTeXML hoc nC CSumLimits DVI symdef newcommand KeyVal baz
% LocalWords:  nx Sumfromto arbitraryn artih activemath
