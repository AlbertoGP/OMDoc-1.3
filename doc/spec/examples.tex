%%%%%%%%%%%%%%%%%%%%%%%%%%%%%%%%%%%%%%%%%%%%%%%%%%%%%%%%%%%%%%%%%%%%%%%%%
% This file is part of the LaTeX sources of the OMDoc 1.3 specification
% Copyright (c) 2006 Michael Kohlhase
% This work is licensed by the Creative Commons Share-Alike license
% see http://creativecommons.org/licenses/by-sa/2.5/ for details
%%%%%%%%%%%%%%%%%%%%%%%%%%%%%%%%%%%%%%%%%%%%%%%%%%%%%%%%%%%%%%%%%%%%%%%%%

\begin{tchapter}[id=examples]{The Listings of the Primer Examples}

We list the full text of the examples discussed in the {\omdoc} primer and
specification. 

\begin{tsection}[id=examples:algebra]{Bourbaki's Algebra  Fragment}

\begin{tsubsection}[id=examples:algebra1]{Top-Level Markup}

\lstinputlisting[nolol,index={DOCTYPE,omdoc,metadata,dc:title,dc:creator,dc:date,dc:description,
  dc:source,dc:type,dc:format,theory,symbol,definition,CMP,omtext,omgroup,example}]
  {../../examples/spec/combined/algebra1.omdoc}
\end{tsubsection}

\begin{tsubsection}[id=examples:algebra2]{Formula Markup}

\lstinputlisting[nolol,index={DOCTYPE,omdoc,metadata,dc:title,dc:creator,dc:date,dc:description,
  dc:source,dc:type,dc:format,theory,symbol,definition,CMP,omtext,omgroup,example,
  presentation,OMOBJ,OMA,OMS,OMBIND,OMBVAR}]
  {../../examples/spec/combined/algebra2.omdoc}
\end{tsubsection}

\begin{tsubsection}[id=examples:algebra3]{Full Formalization}

\lstinputlisting[nolol,index={DOCTYPE,omdoc,metadata,dc:title,dc:creator,dc:date,dc:description,
  dc:source,dc:type,dc:format,theory,symbol,definition,CMP,omtext,omgroup,example,
  presentation,OMOBJ,OMA,OMS,OMBIND,OMBVAR}]
  {../../examples/spec/combined/algebra3.omdoc}
\end{tsubsection}

\begin{tsubsection}[id=examples:background]{The Background Theories}

\lstinputlisting[nolol,index={DOCTYPE,omdoc,metadata,dc:title,dc:creator,dc:date,dc:description,
  dc:source,dc:type,dc:format,theory,symbol,definition,CMP,omtext,omgroup,example,
  presentation,OMOBJ,OMA,OMS,OMBIND,OMBVAR}]
  {../../examples/spec/combined/background.omdoc}
\end{tsubsection}
\end{tsection}

\begin{tsection}[id=examples:arith1]{A Fragment from a Content Dictionary}
\lstinputlisting[nolol,index={DOCTYPE,omdoc,metadata,dc:title,dc:creator,dc:date,dc:description,
  dc:source,dc:type,dc:format,theory,symbol,definition,CMP,omtext,omgroup,example,
  presentation,OMOBJ,OMA,OMS,OMBIND,OMBVAR}]
  {../../examples/spec/combined/arith1.omdoc}
\end{tsection}

\begin{tsection}[id=examples:courseware]{Courseware}
This example comes in two parts, we first list the data-structured document and
then the narrative-structured one.
\lstinputlisting[nolol,index={DOCTYPE,omdoc,metadata,dc:title,dc:creator,dc:date,dc:description,
  dc:source,dc:type,dc:format,theory,symbol,definition,CMP,omtext,omgroup,example,
  presentation,OMOBJ,OMA,OMS,OMBIND,OMBVAR}] {../../examples/spec/combined/15-211-thy.omdoc}

\lstinputlisting[nolol,index={DOCTYPE,omdoc,metadata,dc:title,dc:creator,dc:date,dc:description,
  dc:source,dc:type,dc:format,theory,symbol,definition,CMP,omtext,omgroup,example,
  presentation,OMOBJ,OMA,OMS,OMBIND,OMBVAR}]
  {../../examples/spec/combined/15-211-narrative.omdoc}
\end{tsection}

\begin{tsection}[id=examples:natlist,short=Lists of Natural Numbers]{A Parameterized Theory of Lists of Natural Numbers}
\lstinputlisting[nolol,index={DOCTYPE,omdoc,metadata,dc:title,dc:creator,dc:date,dc:description,
  dc:source,dc:type,dc:format,theory,symbol,definition,CMP,omtext,omgroup,example,
  presentation,OMOBJ,OMA,OMS,OMBIND,OMBVAR}]
  {../../examples/spec/combined/natlist.omdoc}
\end{tsection}
\end{tchapter}
%%% Local Variables: 
%%% mode: latex
%%% TeX-master: "omdoc"
%%% End: 

% LocalWords:  arith dc natlist nolol
