%%%%%%%%%%%%%%%%%%%%%%%%%%%%%%%%%%%%%%%%%%%%%%%%%%%%%%%%%%%%%%%%%%%%%%%%%
% This file is part of the LaTeX sources of the OMDoc 1.3 specification
% Copyright (c) 2006 Michael Kohlhase
% This work is licensed by the Creative Commons Share-Alike license
% see http://creativecommons.org/licenses/by-sa/2.5/ for details
%%%%%%%%%%%%%%%%%%%%%%%%%%%%%%%%%%%%%%%%%%%%%%%%%%%%%%%%%%%%%%%%%%%%%%%%%

\documentclass{article}
\usepackage{url,xmlindex}
\usepackage[hide]{errata}
\usepackage[show]{ed}
\usepackage[a4paper=true,  bookmarks=true, linkcolor=black,
            citecolor=black,urlcolor=black,colorlinks=true,pagecolor=black,
            breaklinks=true, bookmarksopen=true]{hyperref}
\usepackage[today]{svninfo}
\def\omdoc{OMDoc}
\def\mathml{\scshape{MathML}}
\def\openmath{\scshape{OpenMath}}
\def\snippet#1{\texttt{#1}}
\title{Errata to the OMDoc 1.3 Specification}
\author{Michael Kohlhase\\
  Computer Science\\
  Jacobs University Bremen\\
  \url{http://kwarc.info/kohlhase}}

\begin{document}
\maketitle
\begin{abstract}
  This document tracks the errata in the OMDoc 1.3 specification. We
  will keep a corrected version available at
  \url{https://svn.omdoc.org/repos/omdoc/branches/omdoc-1.3/doc/spec/spec.pdf}.
\end{abstract}

\section{Introduction}
The {\omdoc} (Open Mathematical Documents) format is a content markup scheme for
(collections of) mathematical documents including articles, textbooks, interactive books,
and courses.  {\omdoc} also serves as the content language for agent communication of
mathematical services on a mathematical software bus.  The format features a modularized
language design, {\openmath} and {\mathml} for representing mathematical objects, and has
been employed and validated in various applications.

\begin{oldpart}{adapt}The {\omdoc} 1.2 specification has been released as volume 4180 in the Springer Lecture
Notes on Artificial Intelligence (LNAI) series. As with any release, the release of the
specification has brought wider use and this flushes out bugs that went unnoticed before.
These bugs (called errata for paper documents) are tracked in this document, whose newest
version can be found at
\url{https://svn.omdoc.org/repos/omdoc/branches/omdoc-1.2/doc/spec/errata.pdf}. A
version of the OMDoc specification that contains all errata corrections (and markup of
what changed) can be found at
\url{https://svn.omdoc.org/repos/omdoc/branches/omdoc-1.2/doc/spec/spec.pdf}.

In the following we will tabulate the errata in document order. Their location will be
referenced by the section they appear in rather than the page number, since we do not
expect the former to change in the errata correction process.
\end{oldpart}
\newpage
\section{The Errata}
\printerrata{spec}

\newpage
\section{Clarifications}

\subsection{The {\attributeshort{cdbase}} Attribute in {\openmath} and {\omdoc}}

In section {\textbf{13.1.1}} we recap the usage of the {\attributeshort{cdbase}} attribute
on {\openmath} objects as a device to ``disambiguate content dictionaries''. In
particular, {\attribute[ns-elt=om]{cdbase}{OMS}} attributes on {\element[ns-elt=om]{OMS}}
elements can be elided when they can be inherited from parent elements.

In section {\textbf{15.6.2}} we very briefly discuss another space-saving inheritance rule
for {\attribute[ns-elt=om]{cdbse}{OMS}} attributes: {\attribute[ns-elt=om]{cdbse}{OMS}}
attributes can be inherited from {\element{imports}} elements. 

As recent misunderstandings in an implementation show, this inheritance mechanism needs
clarification. 

The general background of this is that on the one hand an {\openmath} symbol (encoded as
an {\element[ns-elt=om]{OMS}} element) is fully identified by a triple: the content
dictionary base, the content dictionary, and the name of the symbol. On the other hand,
{\omdoc} specifies that the visibility of symbols in {\omdoc} documents is governed by
theories (the {\omdoc} counterparts of content dictionaries): a symbol can only be used in
a context that imports the symbol's home theory. Thus we can use the theory context to
disambiguate theories of symbols and no {\element[ns-elt=om]{OMS}} element in an {\omdoc}
document needs to carry an explicit {\attribute[ns-elt=om]{cdbase}{OMS}} attribute.

To compute the content dictionary base of a symbol, we must fist compute its theory
context, which is a partial function from theory names to URIs given by the following set
of rules:
\begin{enumerate}
\item The {\texttt{immediate theory context}} of a theory consists of the theory name (given in the
  {\attribute[ns-attr=xml]{id}{theory}} attribute on the {\element{theory}} element) its
  base URI (as defined in~\cite[section 5.1]{BerFieMas:05}).
\item Let $T$ be a theory that imports theories $S_1,\ldots,S_n$. Furthermore let
  $\sigma_i$ be the theory context of the theories $S_i$, $\iota$ the immediate theory
  context of $T$ and $\pi$ the theory context the parent theory of $T$ if it exists, else
  $\emptyset$. Then the {\texttt{theory context}} $\theta$ of $T$ is defined by
  $\theta\colon=\pi\cup\iota\cup\sigma_1\cup\ldots\cup\sigma_n$, where $\cup$ is the union
  of partial functions. Note that we take $\cup$ to be commutative by making it undefined,
  if its arguments contradict each other.
\end{enumerate}
With this we can define content dictionary base of a symbol $s$ with name $n$
\begin{enumerate}
\item If $T$ is the nearest {\element{theory}} ancestor of $s$ and $T$ has theory context
  $\theta$, then the content dictionary base of $s$ is $\theta(n)$.
\item If $s$ is not contained in a theory and $T$ is the theory referenced by the nearest
  ancestor element of $s$ with a {\attributeshort{theory}} attribute and $\theta$ is the
  theory context of that, then the content dictionary base of $s$ is $\theta(n)$ --- which
  may be undefined if $\theta$ does not supply a content dictionary base for $n$.
\item Otherwise the content dictionary base of $s$ is undefined.
\end{enumerate}
We call an {\omdoc} document $o$ {\textbf{well-scoped}}, iff for any symbol $s$ in $o$,
the content dictionary base is defined. We require that any {\omdoc} document is
well-scoped. In particular, an {\omdoc} application should issue an error, if it reads a
document that is not well-scoped. 

\bibliographystyle{alphahurl}
\bibliography{kwarc}
\end{document}

%%% Local Variables: 
%%% mode: latex
%%% fill-column: 90
%%% TeX-master: t
%%% End: 

% LocalWords:  ps pdf tex LNAI cdbase ns elt cdbse attr xml kwarc
