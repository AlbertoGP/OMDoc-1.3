%%%%%%%%%%%%%%%%%%%%%%%%%%%%%%%%%%%%%%%%%%%%%%%%%%%%%%%%%%%%%%%%%%%%%%%%%
% This file is part of the LaTeX sources of the OMDoc 1.3 specification
% Copyright (c) 2016 Michael Kohlhase.
% Source at https://github.com/KWARC/OMDoc/tree/master/doc/spec
% This work is licensed by the Creative Commons Share-Alike license
% see http://creativecommons.org/licenses/by-sa/2.5/ for details
%%%%%%%%%%%%%%%%%%%%%%%%%%%%%%%%%%%%%%%%%%%%%%%%%%%%%%%%%%%%%%%%%%%%%%%%%

\begin{tchapter}[id=mobj,short=Mathematical Objects]{Mathematical Objects (Module {\MOBJmodule{spec}})}

  A distinguishing feature of mathematics is its ability to represent and manipulate ideas
  and objects in symbolic form as mathematical formulae\index{formula}.  {\omdoc} uses the
  {\openmath} and {\cmathml} formats to represent mathematical formulae and objects.
  Therefore, the {\openmath} standard~\cite{BusCapCar:2oms04} and the {\mathml} 2.0
  recommendation (second edition)~\cite{CarIon:MathML03} are part of this specification.
  We will review {\openmath} objects (top-level element {\element[ns-elt=om]{OMOBJ}}) in
  {\mysecref{openmath}} and {\cmathml} (top-level element {\element[ns-elt=m]{math}}) in
  {\mysecref{cmml}}, and specify an {\omdoc} element for entering mathematical formulae
  (element {\element{legacy}}) in {\mysecref{legacy}}.

\begin{myfig}{mobjtable}{Mathematical Objects in {\omdoc}}
\begin{scriptsize}
\begin{tabular}{|>{\tt}l|>{\tt}p{1.5truecm}|>{\tt}l|>{\tt}l|}\hline
{\rm Element}& \multicolumn{2}{l|}{Attributes\hspace*{2.25cm}} & Content  \\\hline
             & {\rm Required}  & {\rm Optional}     &           \\\hline\hline
 OMOBJ   & id & class, style &  {\rm See {\myfigref{om}}} \\\hline 
 m:math    & & id, xlink:href        & {\rm See {\myfigref{cmml}}} \\\hline
 legacy  & format & xml:id, formalism  &  \#PCDATA \\\hline
\end{tabular}
\end{scriptsize}
\end{myfig}

The recapitulation in the next two sections is not normative, please consult
{\mysecref{math-objects}} for a general introduction and history and the {\openmath}
standard and the {\mathml} 2.0 Recommendation for details and
clarifications.\ednote{MK@MK: MathML 3.0 is out, and thus OM and MathML are equivalent, we
  should discuss this here. }\ednote{MK@MK: discuss the notion of \openmath validity in
  terms of \omdoc content dictionaries. Introduce the notion of a CD catalog as a
  document-format internal way of specifying the cdbase. \openmath cannot have an internal
  one, since it does not transcend the object level. Cite the MathML3.0 spec that allows
  this.}


\begin{tsection}[id=openmath]{OpenMath}
  
  {\openmath} is a markup language for mathematical formulae that concentrates on the
  meaning of formulae building on an extremely simple kernel (markup primitive for
  syntactical forms of content formulae), and adds an extension mechanism for mathematical
  concepts, the {\defemph{content dictionaries}}\twin{content}{dictionary}.  These are
  {\indextoo{machine-readable}} documents that define the meaning of mathematical concepts
  expressed by {\openmath} symbols.  The current released version of the {\openmath}
  standard is {\openmath}2, which incorporates many of the experiences of the last years,
  particularly with embedding {\openmath} into the {\omdoc} format.

  We will only review the {\xml} encoding of {\openmath} objects here, since it is most
  relevant to the {\omdoc} format. All elements of the {\xml} encoding live in the
  {namespace\twin{OpenMath}{namespace}} \url{http://www.openmath.org/OpenMath}, for
  which we traditionally use the {\twintoo{namespace}{prefix}}
  {\snippetin{om:}}.\atwin{OpenMath}{namespace}{URI}

\begin{myfig}{om}{{\openmath} Objects in {\omdoc}}
\begin{scriptsize}
\begin{tabular}{|>{\tt}l|>{\tt}l|>{\tt}l|>{\tt}l|}\hline
{\rm Element}& \multicolumn{2}{l|}{Attributes\hspace*{2.25cm}} & Content  \\\hline
             & {\rm Required}  & {\rm Optional}     &           \\\hline\hline
 OMOBJ     & & id, cdbase, class, style   & \llquote{OMel}? \\\hline
 OMS       & cd, name  & id, cdbase, class, style   &  EMPTY \\\hline
 OMV       & name & id, class, style   &  EMPTY \\\hline
 OMA       & & id, cdbase, class, style   & \llquote{OMel}* \\\hline
 OMBIND    & & id, cdbase, class, style   & \llquote{OMel},OMBVAR,\llquote{OMel} \\\hline
 OMBVAR    & & id, class, style   & (OMV | OMATTR)+ \\\hline
 OMFOREIGN & & id, cdbase, class, style   & ANY \\\hline
 OMATTR    & & id, cdbase, class, style   & \llquote{OMel}\\\hline
 OMATP     & & id, cdbase, class, style   & (OMS, (\llquote{OMel}|OMFOREIGN))+ \\\hline
 OMI       & & id, class, style   &  [0-9]* \\\hline 
 OMB       & & id,  class, style   &  \#PCDATA \\\hline 
 OMF       & & id, class, style, dec, hex &  \#PCDATA \\\hline 
 OME       & & id, class, style   & \llquote{OMel}?\\\hline
 OMR       & href &      & \llquote{OMel}?\\\hline
 \multicolumn{4}{|l|}{where {\llquote{OMel}} is {\tt{(OMS|OMV|OMI|OMB|OMSTR|OMF|OMA|OMBIND|OME|OMATTR)}}}\\\hline
\end{tabular}
\end{scriptsize}
\end{myfig}

\begin{tsubsection}[id=om:core]{The Representational Core of {\openmath}}
The central construct of the {\openmath} is that of an {\defemph{{\openmath}
    object}}\index{object!{\sc OpenMath}}\index{{\sc OpenMath}! object} (represented by
the {\eldef[ns-elt=om]{OMOBJ}} element in the {\xml} encoding), which has a tree-like
representation made up of {\indextoo{application}s} ({\eldef[ns-elt=om]{OMA}}),
{\indextoo{binding structure}s} ({\element[ns-elt=om]{OMBIND}} using
{\element[ns-elt=om]{OMBVAR}} to tag {\twintoo{bound}{variable}s}), {\indextoo{variable}s}
({\eldef[ns-elt=om]{OMV}}), and {\indextoo{symbol}s} ({\eldef[ns-elt=om]{OMS}}).

The {\element[ns-elt=om]{OMA}} element contains representations of the function and its
argument in ``prefix-\twin{prefix}{notation}'' or ``{\twintoo{Polish}{notation}}'',
i.e. the first child is the representation of the function and all the subsequent ones are
representations of the arguments in order.

Objects and concepts that carry meaning independent of the local context (they are called
{\defin{symbol}s} in {\openmath}) are represented as {\element[ns-elt=om]{OMS}} elements,
where the value of the {\attribute[ns-elt=om]{name}{OMS}} attribute gives the name of the
symbol.  The {\attribute[ns-elt=om]{cd}{OMS}} attribute specifies the relevant
{\indextoo{content dictionary}}, a document that defines the meaning of a collection of
symbols including the one referenced by the {\element[ns-elt=om]{OMS}}.  This document can
either be an original {\openmath} content dictionary or an {\omdoc} document that serves
as one (see {\mysubsecref{identifying}} for a discussion).  The optional
{\attributeshort{cdbase}} on an {\element[ns-elt=om]{OMS}} element contains a
{\indextoo{URI}} that can be used to disambiguate the content dictionary.  Alternatively,
the {\attributeshort{cdbase}} attribute can be given on an {\openmath} element that is a
parent to the {\element[ns-elt=om]{OMS}} in question: The {\element[ns-elt=om]{OMS}}
inherits the {\attributeshort{cdbase}} of the nearest ancestor (inducing the usual {\xml}
scoping rules for declarations).

The {\openmath}2 standard proposes the following mechanism for determining a canonical
identifying {\indextoo{URI}} for the symbol declaration referenced by an {\openmath}
symbol of the form {\snippet{<OMS cd="foo" name="bar"/>}} with the
{\attributeshort{cdbase}}-value e.g.  \url{http://www.openmath.org/cd}: it is the URI
reference \url{http://www.openmath.org/cd/foo#bar}, which by convention identifies an
{\element[ns-elt=omcd]{CDDefinition}} element with a child {\element[ns-elt=omcd]{Name}}
whose value is {\snippet{bar}} in a content dictionary resource
\url{http://www.openmath.org/cd/foo.ocd} (see {\mysubsecref{math-markup:openmath}} for a
very brief introduction to {\openmath} content dictionaries). 

Variables are represented as {\element[ns-elt=om]{OMV}} element.  As variables do not
carry a meaning independent of their local content, {\element[ns-elt=om]{OMV}} only
carries a {\attribute[ns-elt=om]{name}{OMV}} attribute (see {\mysecref{sem-var}} for
further discussion).

For instance, the formula $\sin(x)$ would be modeled as an application of the
$\sin$ function (which in turn is represented as an {\openmath} symbol) to a
variable:
\begin{lstlisting}[label=sinx,language=OpenMath,numbers=none,
   index={OMOBJ,OMA,OMV,OMS}]
<OMOBJ xmlns="http://www.openmath.org/OpenMath">
  <OMA cdbase="http://www.openmath.org/cd">
    <OMS cd="transc1" name="sin"/>
    <OMV name="x"/>
  </OMA>
</OMOBJ>   
\end{lstlisting}

In our case, the function $\sin$ is represented as an {\element[ns-elt=om]{OMS}} element
with name {\snippet{sin}} from the {\indextoo{content dictionary}}
{\snippet{transc1}}. The {\element[ns-elt=om]{OMS}} inherits the
{\attributeshort{cdbase}}-value \url{http://www.openmath.org/cd}, which shows that it
comes from the {\openmath} standard collection of content dictionaries from the
{\element[ns-elt=om]{OMA}} element above.  The variable $x$ is represented in an
{\element[ns-elt=om]{OMV}} element with {\attribute[ns-elt=om]{name}{OMV}}-value
{\snippet{x}}.

For the {\eldef[ns-elt=om]{OMBIND}} element consider the following representation of the
formula $\allcdot{x}{\sin(x)\leq\pi}$.
\begin{lstlisting}[label=allxsinx,language=OpenMath,numbers=none,
   index={OMOBJ,OMA,OMV,OMBIND,OMBVAR}]
<OMOBJ cdbase="http://www.openmath.org/cd">
  <OMBIND>
    <OMS cd="quant1" name="forall"/>
    <OMBVAR><OMV name="x"/></OMBVAR>
    <OMA>
      <OMS cd="arith1" name="leq"/>
      <OMA><OMS cd="transc1" name="sin"/><OMV name="x"/></OMA>
      <OMS cd="nums1" name="pi"/>
    </OMA>
  </OMBIND>
</OMOBJ>   
\end{lstlisting}
The {\element[ns-elt=om]{OMBIND}} element has exactly three children, the first one is a
``{\twintoo{binding}{operator}}''\footnote{\label{foot:binding-operator}The binding
  operator must be a symbol which either has the {\indextoo{role}}
  {\attvalshort{binder}{role}} assigned by the {\openmath} content dictionary
  (see~\cite{BusCapCar:2oms04} for details) or the symbol declaration in the {\omdoc}
  content dictionary must have the value {\attval{binder}{role}{symbol}} for the attribute
  {\attribute{role}{symbol}} (see {\mysubsecref{symbol-dec}}).} --- in this case the
universal quantifier, the second one is a list of {\twintoo{bound}{variable}s} that must
be encapsulated in an {\eldef[ns-elt=om]{OMBVAR}} element, and the third is the
{\indextoo{body}} of the binding object, in which the bound variables can be used.
{\openmath} uses the {\element[ns-elt=om]{OMBIND}} element to unambiguously specify the
scope of bound variables in expressions: the bound variables in the
{\element[ns-elt=om]{OMBVAR}} element can be used only inside the mother
{\element[ns-elt=om]{OMBIND}} element, moreover they can be systematically
renamed\twin{variable}{renaming} without changing the meaning of the binding
expression. As a consequence, bound variables in the scope of an
{\element[ns-elt=om]{OMBIND}} are distinct as {\openmath} objects from any variables
outside it, even if they share a name.

{\openmath} offers an element for annotating (parts of) formulae with external information
(e.g. {\mathml} or {\LaTeX} presentation): the {\eldef[ns-elt=om]{OMATTR}} element that
pairs an {\openmath} object with an attribute-value list. To annotate an {\openmath}
object, it is embedded as the second child in an {\element[ns-elt=om]{OMATTR}}
element. The attribute-value list is specified by children of the preceding
{\eldef[ns-elt=om]{OMATP}} ({\underline{At}}tribute value {\underline{P}}air) element,
which has an even number of children: children at odd positions must be
{\element[ns-elt=om]{OMS}} (specifying the attribute, they are called {\defin{key}s} or
{\defin{feature}s})\footnote{There are two kinds of keys in {\openmath} distinguished
  according to the {\attribute{role}{symbol}} value on their {\element{symbol}}
  declaration in the {\indextoo{content}{dictionary}}:
  {\attval{attribution}{role}{symbol}} specifies that this attribute value pair may be
  ignored by an application, so it should be used for information which does not change
  the meaning of the attributed {\openmath} object. The {\attribute{role}{symbol}} is used
  for keys that modify the meaning of the attributed {\openmath} object and thus cannot be
  ignored by an application.}, and children at even positions are the {\defin{value}s} of
the keys specified by their immediately preceding siblings. In the {\openmath} fragment in
{\mylstref{omattr}} the expression $x+\pi$ is annotated with an alternative representation
and a color.  {\Mylstref{complex-type-om}} has a more complex one involving types.

\begin{lstlisting}[language=OpenMath,label=lst:omattr,mathescape,
                   caption={Associating Alternate Representations with an
                   {\openmath} Object},
                   numbers=none,index={OMATTR,OMATP}]
<OMATTR>
  <OMATP>
    <OMS cd="alt-rep" name="ascii"/>
    <OMSTR>(x+1)</OMSTR>
    <OMS cd="alt-rep" name="svg"/>
    <OMFOREIGN encoding="application/svg+xml">
      <svg xmlns='http://www.w3.org/2000/svg'>$\ldots$</svg>
    </OMFOREIGN>
    <OMS cd="pres" name="color"/>
    <OMS cd="pres" name="red"/>
  </OMATP>
  <OMA>
    <OMS cd="arith1" name="plus"/>
    <OMV name="x"/>
    <OMS cd="nums1" name="pi"/>
  </OMA>
</OMATTR>
\end{lstlisting}

A special application of the {\element[ns-elt=om]{OMATTR}} element is associating
non-{\sc{Open\-Math}} objects with {\openmath} objects. For this, {\openmath}2 allows to
use an {\eldef[ns-elt=om]{OMFOREIGN}} element in the even positions of an
{\element[ns-elt=om]{OMATP}}. This element can be used to hold arbitrary {\xml} content
(in our example above SVG: Scalable Vector Graphics~\cite{W3C:svg02}), its required
{\attribute[ns-elt=om]{encoding}{OMFOREIGN}} attribute specifies the format of the
content.  We recommend a {\twintoo{MIME}{type}}~\cite{FreBor:MIME96} (see
{\mysecref{pres-bound}} for an application).
\end{tsubsection}
  
\begin{tsubsection}[id=om:error]{Programming Extensions of {\openmath} Objects}

  For representing objects in {\twintoo{computer algebra}{system}s} {\openmath} also
  provides other basic data types: {\eldef[ns-elt=om]{OMI}} for {\indextoo{integer}s},
  {\eldef[ns-elt=om]{OMB}} for {\indextoo{byte array}s}, {\eldef[ns-elt=om]{OMSTR}} for
  {\indextoo{string}s}, and {\eldef[ns-elt=om]{OMF}} for floating point numbers. These do
  not play a large role in the context of {\omdoc}, so we refer the reader to the
  {\openmath} standard~\cite{BusCapCar:2oms04} for details.


  The {\eldef[ns-elt=om]{OME}} element is used for {\atwintoo{in-place}{error}{markup}} in
  {\openmath} objects, it can be used almost everywhere in {\openmath} elements. It has
  two children; the first one is an {\twintoo{error}{operator}}\footnote{An error operator
    is like a {\twintoo{binding}{operator}} in footnote~\ref{foot:binding-operator}, only
    the symbol has role {\attval{error}{role}{symbol}}.}, i.e. an {\openmath} symbol that
  specifies the kind of error, and the second one is the faulty {\openmath} object
  fragment. Note that since the whole object must be a valid {\openmath} object, the
  second child must be a well-formed {\openmath} object fragment. As a consequence, the
  {\element[ns-elt=om]{OME}} element can only be used for ``{\twintoo{semantic}{error}s}''
  like non-existing content dictionaries, out-of-bounds errors, etc.
  {\xml}-well-formedness and DTD-validity errors will have to be handled by the {\xml}
  tools involved. In the following example, we have marked up two errors in a faulty
  representation of $\sin(\pi)$.  The outer error flags an arity violation (the function
  $\sin$ only allows one argument), and the inner one flags the typo in the representation
  of the constant $\pi$ (we used the name {\snippet{po}} instead of {\snippet{pi}}).

\begin{lstlisting}[label=ome,language=OpenMath,numbers=none,index={OME}]
<OME>
  <OMS cd="type-error" name="arity-violation"/>
  <OMA>
    <OMS cd="transc1" name="sin"/>
    <OME>
      <OMS cd="error" name="unexpected_symbol"/>
      <OMS cd="nums1" name="po"/>
    </OME>
    <OMV name="x"/>
  </OMA>
</OME>
\end{lstlisting}
  As we can see in this example, errors can be nested to encode multiple faults found by
  an {\openmath} application.
\end{tsubsection}

\begin{tsubsection}[id=om:structure-sharing]{Structure Sharing in {\openmath}}

  As we have seen above, {\openmath} objects are essentially trees, where the leaves are
  symbols or variables. In many applications mathematical objects can grow to be very
  large, so that more space-efficient representations are needed. Therefore, {\openmath}2
  supports {\twintoo{structure}{sharing}}\footnote{Structure sharing is a well-known
    technique in computer science that tries to gain space efficiency in algorithms by
    re-using data structures that have already been created by pointing to them rather
    than copying.} in {\openmath} objects. In {\myfigref{sharing}} we have contrasted the
  tree representation of the object $1+1+1+1+1+1+1+1$ with the structure-shared one, which
  represents the formula as a {\atwintoo{directed}{acyclic}{graph}} (\indextoo{DAG}). As
  any DAG can be exploded\twin{explosion}{DAG} into a tree by recursively copying all
  sub-graphs that have more than one incoming graph edge, DAGs can conserve space by
  {\twintoo{structure}{sharing}}. In fact the tree on the left in {\myfigref{sharing}} is
  exponentially larger than the corresponding DAG on the right.

\begin{myfig}{sharing}{Structure Sharing by Directed Acyclic Graphs}
  \begin{tikzpicture}
    \node (u) at (-1,0) {$\cdot$};
    \node (m) at (-1,1.5) {$d$};
    \node (o) at (-1,3) {$\cdot$}; 
    \draw[->] (m) -- (u);
    \draw[->] (m) -- (o);
    
    \node (a1) at (0,0) {1}; 
    \node (a2) at (1,0) {1};
    \node (a3) at (2,0) {1}; 
    \node (a4) at (3,0) {1};
    \node (a5) at (4,0) {1}; 
    \node (a6) at (5,0) {1};
    \node (a7) at (6,0) {1}; 
    \node (a8) at (7,0) {1};
    
    \node (f1) at (0.5,1) {+};
    \node (f2) at (2.5,1) {+};
    \node (f3) at (4.5,1) {+}; 
    \node (f4) at (6.5,1) {+};
    \node (f5) at (2,2) {+}; 
    \node (f6) at (5,2) {+};
    \node (f7) at (3.5,3) {+}; 
    
    \draw (f1) -- (a1);
    \draw (f1) -- (a2);
    \draw (f2) -- (a3);
    \draw (f2) -- (a4); 
    \draw (f3) -- (a5);
    \draw (f3) -- (a6);
    \draw (f4) -- (a7);
    \draw (f4) -- (a8); 
    \draw (f5) -- (f1);
    \draw (f5) -- (f2);
    \draw (f6) -- (f3);
    \draw (f6) -- (f4); 
    \draw (f7) -- (f5);
    \draw (f7) -- (f6);
    
    \node (a) at (9,0) {1}; 
    \node (F1) at (9,1) {+};
    \node (F2) at (9,2) {+}; 
    \node (F3) at (9,3) {+};
    \draw (a) .. controls (9.2,.4) and (9.2,.6)  .. (F1);
    \draw (a) .. controls (8.8,.4) and (8.8,.6)  .. (F1);
    \draw (F1) .. controls (9.2,1.4) and (9.2,1.6)  .. (F2);
    \draw (F1) .. controls (8.8,1.4) and (8.8,1.6)  .. (F2);
    \draw (F2) .. controls (9.2,2.4) and (9.2,2.6)  .. (F3);
    \draw (F2) .. controls (8.8,2.4) and (8.8,2.6)  .. (F3);

    \node (t)  at (3.5,4)  {Tree};
    \node (d)  at (9,4)    {DAG};
    \node (nt) at (3.5,-1) {$2^d-1$ nodes}; 
    \node (nd) at (9,-1)   {$d$ nodes};
  \end{tikzpicture}
\end{myfig}
To support DAG structures, {\openmath}2 provides the (optional) attribute
{\attributeshortcomment{id}{in {\sc{OpenMath}} objects}} on all {\openmath} objects and an
element {\eldef[ns-elt=om]{OMR}}\footnote{{\openmath}1 and {\omdocv{1.0}} did now know
  structure sharing, {\omdocv{1.1}} added {\attributeshort{xref}} attributes to the
  {\openmath} elements {\element[ns-elt=om]{OMOBJ}}, {\element[ns-elt=om]{OMA}},
  {\element[ns-elt=om]{OMBIND}} and
  {\element[ns-elt=om]{OMATTR}}\index{openmath@{\sc{OpenMath}} elements!  extra attributes
    {\snippet{id}} and {\snippet{xref}}} instead of {\element[ns-elt=om]{OMR}}
  elements. This usage is deprecated in {\omdocv{1.2}}, in favor of the
  {\element[ns-elt=om]{OMR}}-based solution from the {\openmath}2 standard. Obviously,
  both representations are equivalent, and a transformation from
  {\attributeshort{xref}}-based mechanism to the {\element[ns-elt=om]{OMR}}-based one is
  immediate.} for the purpose of cross-referencing\index{cross-reference}. The
{\element[ns-elt=om]{OMR}} element is empty and has the required attribute
{\attribute[ns-elt=om]{href}{OMR}}; The {\openmath} element represented by this
{\element[ns-elt=om]{OMR}} element is a copy of the {\openmath} element pointed to in the
{\attribute[ns-elt=om]{href}{OMR}} attribute.  Note that the representation of the
{\element[ns-elt=om]{OMR}} element is {\em{structurally equal}}, but not identical to the
element it points to.

Using the {\element[ns-elt=om]{OMR}} element, we can represent the {\openmath} objects in
{\myfigref{sharing}} as the {\xml} representations in {\myfigref{explosion}}.

\setbox0=\hbox{\begin{minipage}{5.2cm}
\begin{lstlisting}[label=exploded,language=OpenMath,numbers=none,frame=none,
   index={OMOBJ,OMA,OMV,href,OMR,id}]
<OMOBJ>                        
  <OMA>                        
    <OMS cd="nat" name="plus"/>               
    <OMA>                      
      <OMS cd="nat" name="plus"/>             
      <OMA>                    
        <OMS cd="nat" name="plus"/>           
        <OMI>1</OMI>
        <OMI>1</OMI>
      </OMA>                    
      <OMA>                     
        <OMS cd="nat" name="plus"/>
        <OMI>1</OMI>
        <OMI>1</OMI>
      </OMA>
    </OMA>                      
    <OMA>                       
      <OMS cd="nat" name="plus"/>
      <OMA>                                           
        <OMS cd="nat" name="plus"/>
        <OMI>1</OMI>
        <OMI>1</OMI>
      </OMA>   
      <OMA>
        <OMS cd="nat" name="plus"/>
        <OMI>1</OMI>
        <OMI>1</OMI>
      </OMA>
    </OMA>                      
  </OMA>                        
</OMOBJ>                        
\end{lstlisting}
\end{minipage}}

\setbox1=\hbox{\begin{minipage}{5.2cm}
\begin{lstlisting}[label=shared,language=OpenMath,
     index={OMOBJ,OMA,OMV},numbers=none,frame=none]
<OMOBJ>
  <OMA>
    <OMS cd="nat" name="plus"/>
    <OMA id="t1">
      <OMS cd="nat" name="plus"/>
      <OMA id="t11">
        <OMS cd="nat" name="plus"/>
        <OMI>1</OMI>
        <OMI>1</OMI>
      </OMA>
      <OMR href="#t11"/>




    </OMA>
    <OMR href="#t1"/>












  </OMA>
</OMOBJ>
\end{lstlisting}
  \end{minipage}}

\begin{myfig}{explosion}{The {\openmath} Objects from {\myfigref{sharing}} in {\xml} Encoding}
\begin{tabular}{|c|c|}\hline
Shared & Exploded \\\hline\hline
\box1 & \box0\\\hline
\end{tabular}
\end{myfig}

To ensure that the {\xml} representations actually correspond to directed
acyclic\atwin{directed}{acyclic}{graph} graphs, the occurrences of the
{\element[ns-elt=om]{OMR}} must obey the global acyclicity constraint below, where we say
that an {\openmath} element {\defin{dominate}s} all its children and all elements they
dominate; The {\element[ns-elt=om]{OMR}} also dominates its {\defin{target}}\footnote{The
  target of an {\openmath} element with an {\attributeshortcomment{id}{in {\sc{OpenMath}}
      objects}} attribute is defined analogously}, i.e. the element that carries the
{\attributeshortcomment{id}{in {\sc{OpenMath}} objects}} attribute pointed to by the
{\attribute[ns-elt=om]{href}{OMR}} attribute.  For instance, in the representation in
{\myfigref{explosion}} the {\element[ns-elt=om]{OMA}} element with {\snippet{xml:id="t1"}}
and also the second {\element[ns-elt=om]{OMA}} element dominate the
{\element[ns-elt=om]{OMA}} element with {\snippet{xml:id="t11"}}.
\begin{quote}
  {\bf {\openmath} Acyclicity Constraint}:\\
  An OpenMath element may not dominate itself.
\end{quote}
\begin{lstlisting}[language=OpenMath,numbers=none,label=lst:chained-fraction,
  caption={A Simple Cycle},
  index={OMOBJ,OMA,OMV}]
<OMOBJ>
  <OMA id="foo">
    <OMS cd="nat" name="divide"/>
    <OMI>1</OMI>
    <OMA><OMS cd="nat" name="plus"/>
      <OMI>1</OMI>
      <OMR href="#foo"/>
    </OMA> 
  </OMA>
</OMOBJ>
\end{lstlisting}
In {\mylstref{chained-fraction}} the {\element[ns-elt=om]{OMA}} element with {\snippet{xml:id="foo"}}
dominates its third child, which dominates the {\element[ns-elt=om]{OMR}} with
{\snippet{href="foo"}}, which dominates its target: the {\element[ns-elt=om]{OMA}} element
with {\snippet{xml:id="foo"}}. So by transitivity, this element dominates itself, and by
the acyclicity constraint, it is not the {\xml} representation of an {\openmath}
object.  Even though it could be given the interpretation of the continued
fraction
\[1\over{1 + {1\over {1+\cdots}}}\] this would correspond to an infinite tree of
applications, which is not admitted by the {\openmath} standard. Note that the acyclicity
constraint is not restricted to such simple cases, as the example in {\mylstref{2cycle}}
shows.  Here, the {\element[ns-elt=om]{OMA}} with {\snippet{xml:id="bar"}} dominates its
third child, the {\element[ns-elt=om]{OMR}} element with {\snippet{href="baz"}}, which
dominates its target {\element[ns-elt=om]{OMA}} with {\snippet{xml:id="baz"}}, which in
turn dominates its third child, the {\element[ns-elt=om]{OMR}} with
{\snippet{href="bar"}}, this finally dominates its target, the original
{\element[ns-elt=om]{OMA}} element with {\snippet{xml:id="bar"}}. So again, this pair of
{\openmath} objects violates the acyclicity constraint and is not the {\xml} encoding of
an {\openmath} object.
\begin{lstlisting}[language=OpenMath,numbers=none,label=lst:2cycle,
  caption={A Cycle of Order Two},index={OMOBJ,OMA,OMV}]
<OMOBJ>                                   <OMOBJ>
  <OMA id="bar">                            <OMA id="baz">
    <OMS cd="nat" name="plus"/>               <OMS cd="nat" name="plus"/>
    <OMI>1</OMI>                              <OMI>1</OMI>
    <OMR href="#baz"/>                        <OMR href="#bar"/>
  </OMA>                                    </OMA>
</OMOBJ>                                  </OMOBJ>
\end{lstlisting}
\end{tsubsection}
\end{tsection}

\begin{tsection}[id=cmml]{Content MathML}
  
  {\cmathml} is a content markup format that represents the abstract structure of formulae
  in trees of logical sub-expressions much like {\openmath}.  However, in contrast to that,
  {\cmathml} provides a lot of primitive tokens and constructor elements for the
  {\indextoo{K-14}} fragment of mathematics ({\indextoo{Kindergarten}} to $14^{th}$ grade
  (i.e. undergraduate college level)).

  The current released version of the {\mathml} recommendation is the second edition of
  {\mathml} 2.0~\cite{CarIon:MathML03}, a maintenance release for the {\mathml} 2.0
  recommendation~\cite{CarIon:MathML01} that cleans up many semantic issues in the content
  {\mathml} part. We will now review those parts of {\mathml} 2.0 that are relevant to
  {\omdoc}; for the full story see~\cite{CarIon:MathML03}.

  Even though {\omdoc} allows full {\cmathml}, we will advocate the use of the {\cmathml}
  fragment described in this section, which is largely isomorphic to {\openmath}
  (see~\mysubsecref{omvscmml} for a discussion).

\begin{myfig}{cmml}{{\cmathml} in {\omdoc}}
\begin{scriptsize}
\begin{tabular}{|>{\tt}l|>{\tt}l|>{\tt}p{2truecm}|>{\tt}p{4truecm}|}\hline
{\rm Element}& \multicolumn{2}{l|}{Attributes\hspace*{2.25cm}}  & Content  \\\hline
              & {\rm Required}  & {\rm Optional}     &            \\\hline\hline
 m:math       & & id, xlink:href                   & \llquote{CMel}+\\\hline
 m:apply      & & id, xlink:href                   & m:bvar?,\llquote{CMel}*\\\hline
 m:csymbol    & definitionURL  & id, xlink:href    & m:EMPTY \\\hline
 m:ci         & & id, xlink:href                   & \#PCDATA \\\hline       
 m:cn         & & id, xlink:href                   & ([0-9]|,|.)(*|e([0-9]|,|.)*)?\\\hline       
 m:bvar       & & id, xlink:href                   & m:ci|m:semantics\\\hline
 m:semantics  & & id, xlink:href, definitionURL 
                & \llquote{CMel},(m:annotation | m:annotation-xml)*\\\hline
 m:annotation & & definitionURL, encoding      & \#PCDATA \\\hline
 m:annotation-xml & & definitionURL, encoding      & ANY \\\hline
 \multicolumn{4}{|l|}{where {\llquote{CMel}} is 
     {\tt{m:apply|m:csymbol|m:ci|m:cn|m:semantics}}}\\\hline
\end{tabular}
\end{scriptsize}
\end{myfig}

\begin{tsubsection}[id=mathml-core]{The Representational Core of {\cmathml}}

  The top-level element of {\mathml} is the {\eldef[ns-elt=m]{math}}\footnote{For DTD
    validation {\omdoc} uses the {\twintoo{namespace}{prefix}} ``{\snippetin{m:}}'' for
    {\mathml} elements, since the {\omdoc} DTD needs to include the {\mathml} DTD with an
    {\atwintoo{explicit}{namespace}{prefix}}, as both {\mathml} and {\omdoc} have a
    {\element{selector}} element that would clash otherwise (DTDs are not
    {\indextoo{namespace-aware}}).} element, see {\myfigref{om-commutativity}} for an
  example. Like {\openmath}, {\cmathml} organizes the mathematical objects into a
  functional tree.  The basic objects ({\mathml} calls them {\twintoo{token}{element}s})
  are
  \begin{description}
  \item[{\bf\indextoo{identifier}s}] (element {\eldef[ns-elt=m]{ci}}) corresponding to
    {\indextoo{variable}s}. The content of the {\element[ns-elt=m]{ci}} element is
    arbitrary {\pmathml}, used as the name of the identifier.
  \item[{\bf\indextoo{number}s}] (element {\eldef[ns-elt=m]{cn}}) for number
    expressions. The attribute {\attribute[ns-elt=m]{type}{cn}} can be used to specify the
    mathematical type of the number, e.g. {\snippet{complex}}, {\snippet{real}}, or
    {\snippet{integer}}. The content of the {\element[ns-elt=m]{cn}} element is
    interpreted as the value of the number expression.
  \item[{\bf\indextoo{symbol}s}] (element {\eldef[ns-elt=m]{csymbol}}) for arbitrary
    symbols.  Their meaning is determined by a {\snippetin{definitionURL}} attribute that
    is a {\twintoo{URI}{reference}} that points to a symbol declaration in a defining
    document. The content of the {\element[ns-elt=m]{csymbol}} element is a {\pmathml}
    representation that used to depict the symbol.
  \end{description}
Apart from these generic elements, {\cmathml} provides a set of about 80 empty
content elements that stand for objects, functions, relations, and constructors
from various basic mathematic fields.

The {\eldef[ns-elt=m]{apply}} element does double duty in {\cmathml}: it is not only used
to mark up {\indextoo{application}s}, but also represents {\indextoo{binding}} structures
if it has an {\eldef[ns-elt=m]{bvar}} child; see {\myfigref{om-commutativity}} below for a
use case in a universal quantifier.

The {\eldef[ns-elt=m]{semantics}} element provides a way to annotate {\cmathml} elements
with arbitrary information. The first child of the {\element[ns-elt=m]{semantics}} element
is annotated with the information in the {\eldef[ns-elt=m]{annotation-xml}} (for
{\xml}-based information) and {\eldef[ns-elt=m]{annotation}} (for other information)
elements that follow it. These elements carry
{\attribute[ns-elt=m]{definitionURL}{annotation}} attributes that point to a
``definition'' of the kind of information provided by them. The optional
{\attribute[ns-elt=m]{encoding}{annotation}} is a string that describes the format of the
content.
\end{tsubsection}

\begin{tsubsection}[id=omvscmml]{OpenMath vs. Content MathML}

  {\openmath} and {\mathml} are well-integrated; there are semantics-preserving converters
  between the two formats. {\mathml} supports the {\element[ns-elt=m]{semantics}} element,
  that can be used to annotate {\mathml} presentations of mathematical objects with their
  {\openmath} encoding. Analogously, {\openmath} supports the {\snippet{presentation}}
  symbol in the {\element[ns-elt=om]{OMATTR}} element, that can be used for annotating
  with {\mathml} presentation. {\openmath} is the designated extension mechanism for
  {\mathml} beyond {\twintoo{K-14}{mathematics}}: Any symbol outside can be encoded as a
  {\element[ns-elt=m]{csymbol}} element, whose
  {\attribute[ns-elt=m]{definitionURL}{csymbol}} attribute points to the {\openmath} CD
  that defines the meaning of the symbol. Moreover all of the {\mathml} content elements
  have counterparts in the {\openmath} core content dictionaries\index{content
    dictionary}~\cite{URL:omcd-core}. For the purposes of {\omdoc}, we will consider the
  various representations following four representations of a content symbol in
  {\myfigref{content-symbols}} as equivalent.  Note that the URI in the
  {\attribute[ns-elt=m]{definitionURL}{csymbol}} attribute does not point to a specific
  file, but rather uses its base name for the reference.  This allows a {\mathml} (or
  {\omdoc}) application to select the format most suitable for it.


  \begin{myfig}{content-symbols}{Four equivalent Representations of a Content Symbol}
  \begin{tabular}{|p{11cm}|}\hline
    {\footnotesize\snippet{<m:plus/>}}  \\\hline 
    {\cmathml} token element \\\hline\hline
    {\footnotesize\snippet{<m:plus definitionURL="http://www.openmath.org/cd/arith1\#plus"/>}}\\\hline
    {\cmathml} token element with explicit pointer\\\hline\hline
    {\footnotesize\snippet{<m:csymbol definitionURL="http://www.openmath.org/cd/arith1\#plus"/>}}\\\hline
    empty {\cmathml} {\element[ns-elt=m]{csymbol}} \\\hline\hline
    {\footnotesize\snippet{<m:csymbol
        definitionURL="http://www.openmath.org/cd/arith1\#plus">}}\\
    \hspace{2ex}{\snippet{<m:mo>+</m:mo>}}\\
   {\snippet{</m:csymbol>}}\\\hline
    {\cmathml} {\element[ns-elt=m]{csymbol}} with presentation\\\hline\hline
    {\footnotesize\snippet{<OMS cdbase="http://www.openmath.org/cd" cd="arith1" name="plus"/>}}  \\\hline
    {\openmath} symbol\\\hline
  \end{tabular}
\end{myfig}

\setbox0=\hbox{\begin{minipage}{5.3cm}\def\baselinestretch{.975}
\begin{lstlisting}[label=omvsmom,language=OpenMath,frame=none,numbers=none,
    index={OMOBJ,OMBIND,OMS,OMBVAR,OMV,OMATTR,OMATP}]
<OMOBJ>                            
 <OMBIND>                          
  <OMS cd="quant1" name="forall"/> 
  <OMBVAR>                         
   <OMATTR>                        
    <OMATP>                        
     <OMS cd="sts" name="type"/>   
     <OMS cd="setname1" name="R"/>  
    </OMATP>                       
    <OMV name="a"/>                

   </OMATTR>                        
   <OMATTR>                        
    <OMATP>                        
     <OMS cd="sts" name="type"/>   
     <OMS cd="setname1" name="R"/>  
    </OMATP>                       

    <OMV name="b"/>                
   </OMATTR>                        
  </OMBVAR>                        
   <OMA>                           
    <OMS cd="relation" name="eq"/> 
    <OMA>                          
     <OMS cd="arith1" name="plus"/>
     <OMV name="a"/>               
     <OMV name="b"/>               
    </OMA>                         
    <OMA>                          
     <OMS cd="arith1" name="plus"/>
     <OMV name="b"/>               
     <OMV name="a"/>               
    </OMA>                         
  </OMA>                           
 </OMBIND>                         
</OMOBJ>
\end{lstlisting}
\end{minipage}}
\setbox1=\hbox{\scriptsize\begin{minipage}{4.3cm}
 \begin{lstlisting}[label=omvsmm,language=MathML,frame=none,numbers=none,
     index={math,apply,forall,bvar,ci,eq,plus}]
<m:math>
 <m:apply>  
  <m:forall/>
  <m:bvar> 





   <m:ci type="real">a</m:ci>
  </m:bvar>





  <m:bvar>
   <m:ci type="real">b</m:ci>

  </m:bvar>
  <m:apply>
   <m:eq/>
   <m:apply>
    <m:plus/>
    <m:ci type="real">a</m:ci>
    <m:ci type="real">b</m:ci>
   </m:apply>
   <m:apply>
    <m:plus/>
    <m:ci type="real">b</m:ci>
    <m:ci type="real">a</m:ci>
   </m:apply>
  </m:apply>
 </m:apply>
</m:math>
\end{lstlisting}
\end{minipage}}
\begin{myfig}{om-commutativity}{{\openmath} vs. C-{\mathml} for Commutativity}
\begin{tabular}{cc}
  {\large\openmath}  &  {\large\mathml}\\
  \fbox{\box0} & \fbox{\box1}
\end{tabular}
\end{myfig}
In {\myfigref{om-commutativity}} we have put the {\openmath} and content {\mathml}
encoding of the law of commutativity for the real numbers side by side to show the
similarities and differences. There is an obvious line-by-line similarity for the
tree constructors and token elements. The main difference is the treatment of
types and variables.
\end{tsubsection}
\end{tsection}

\begin{tsection}[id=mobj:types]{Representing Types in {\cmathml} and {\openmath}}


  Types\index{type} are representations of certain simple sets that are treated specially
  in (human or mechanical) {\twintoo{reasoning}{process}es}. In typed representations
  variables and constants are usually associated with types to support more guided
  reasoning processes. Types are structurally like mathematical objects (i.e. arbitrary
  complex trees). Since types are ubiquitous in representations of mathematics, we will
  briefly review the best practices for representing them in {\omdoc}.

  {\mathml} supplies the {\attributeshortcomment{type}{on {\sc{MathML}} objects}}
  attribute to specify types that can be taken from an open-ended list of type names.
  {\openmath} uses the {\element[ns-elt=om]{OMATTR}} element to associate a type (in this
  case the set of real numbers as specified in the {\snippetin{setname1}} content
  dictionary) with the variable, using the {\twintoo{feature}{symbol}} {\snippetin{type}}
  from the {\snippetin{sts}} content dictionary. This mechanism is much more heavy-weight
  in our special case, but also more expressive: it allows to use arbitrary content
  expressions for types, which is necessary if we were to assign e.g. the type
  $(\RR\to\RR)\to(\RR\to\RR)$ for functionals on the real numbers. In such cases, the
  second edition of the {\mathml}2 Recommendation advises a construction using the
  {\element[ns-elt=m]{semantics}} element (see~\cite{DevKoh:stm03} for details).
  {\Mylstsref{complex-type-om}{complex-type-mathml}} show the realizations of a
  quantification over a variable of functional type in both formats.

\begin{lstlisting}[language=OpenMath,label=lst:complex-type-om,,mathescape,
    caption={A Complex Type in {\openmath}},
    index={OMOBJ,OMBIND,OMS,OMBVAR,OMV,OMATTR,OMATP}]
<OMOBJ>                            
  <OMBIND>                          
    <OMS cd="quant1" name="forall"/> 
    <OMBVAR>                         
      <OMATTR>                        
        <OMATP>                        
          <OMS cd="sts" name="type"/>
          <OMA><OMS cd="sts" name="mapsto"/>   
            <OMA><OMS cd="sts" name="mapsto"/>   
              <OMS cd="setname1" name="R"/>  
              <OMS cd="setname1" name="R"/>
            </OMA>
            <OMA><OMS cd="sts" name="mapsto"/>   
              <OMS cd="setname1" name="R"/>  
              <OMS cd="setname1" name="R"/>
            </OMA>
          </OMA>  
        </OMATP>                       
        <OMV name="F"/>                
      </OMATTR>                        
    </OMBVAR>                        
    $\ldots$
  </OMBIND>                         
</OMOBJ>
\end{lstlisting}

Note that we have essentially used the same URI (to the {\snippetin{sts}} content
dictionary) to identify the fact that the annotation to the variable is a type (in
a particular type system).

 \begin{lstlisting}[language=MathML,label=lst:complex-type-mathml,mathescape,
     caption={A Complex Type in {\cmathml}},
     index={math,apply,forall,bvar,ci,csymbol}]
<m:math>
  <m:apply>  
    <m:forall/>
    <m:bvar>
      <m:semantics>
        <m:ci>F</m:ci>
        <m:annotation-xml definitionURL="http://www.openmath.org/cd/sts#type">
          <m:apply>
            <m:csymbol definitionURL="http://www.openmath.org/cd/sts#mapsto"/>
            <m:apply>
              <m:csymbol definitionURL="http://www.openmath.org/cd/sts#mapsto"/>
              <m:csymbol definitionURL="http://www.openmath.org/cd/setname1#real"/>
              <m:csymbol definitionURL="http://www.openmath.org/cd/setname1#real"/>
            </m:apply>        
            <m:apply>
              <m:csymbol definitionURL="http://www.openmath.org/cd/sts#mapsto"/>
              <m:csymbol definitionURL="http://www.openmath.org/cd/setname1#real"/>
              <m:csymbol definitionURL="http://www.openmath.org/cd/setname1#real"/>
            </m:apply>        
          </m:apply>
        </m:annotation-xml>
      </m:semantics>
    </m:bvar>
    $\ldots$
  </m:apply>
</m:math>
 \end{lstlisting}
\end{tsection}

\begin{tsection}[id=sem-var,short=Semantics of Variables]{The Semantics of Variables in
    {\openmath} and {\cmathml}}
 
  A more subtle, but nonetheless crucial difference between {\openmath} and {\mathml} is
  the handling of variables, symbols, their names, and equality conditions.  {\openmath}
  uses the {\attribute[ns-elt=om]{name}{OMV, om:OMS}} attribute to identify a variable or
  symbol, and delegates the presentation of its name to other methods such as style
  sheets. As a consequence, the elements {\element[ns-elt=om]{OMS}} and
  {\element[ns-elt=om]{OMV}} are empty, and we have to understand the value of the
  {\attribute[ns-elt=om]{name}{OMV, om:OMS}} attribute as a {\indextoo{pointer}} to a
  defining occurrence. In case of symbols, this is the symbol declaration in the
  {\twintoo{content}{dictionary}} identified in the {\attribute[ns-elt=om]{cd}{OMS}}
  attribute. A symbol {\snippet{<OMS cd="\llquote{$cd_1$}" name="\llquote{$name_1$}"/>}}
  is equal to {\snippet{<OMS cd="\llquote{$cd_2$}" name="\llquote{$name_2$}"/>}}, iff
  {\llquote{$cd_1$}=\llquote{$cd_2$}} and {\llquote{$name_1$}=\llquote{$name_2$}} as
  {\xml} simple names.  In case of variables this is more difficult: if the variable is
  bound\twin{bound}{variable} by an {\element[ns-elt=om]{OMBIND}} element\footnote{We say
    that an {\element[ns-elt=om]{OMBIND}} element {\defemph{binds}}\index{binding} an
    {\openmath} variable {\snippet{<OMV name="x"/>}}, iff this
    {\element[ns-elt=om]{OMBIND}} element is the nearest one, such that {\snippet{<OMV
        name="x"/>}} occurs in (second child of the {\element[ns-elt=om]{OMATTR}} element
    in) the {\element[ns-elt=om]{OMBVAR}} child (this is the
    {\twindef{defining}{occurrence}} of {\snippet{<OMV name="x"/>}} here).}, then we
  interpret all the variables {\snippet{<OMV name="x"/>}} in the
  {\element[ns-elt=om]{OMBIND}} element as equal and different from any variables
  {\snippet{<OMV name="x"/>}} outside. In fact the {\openmath} standard states that bound
  variables can be renamed\twin{renaming}{variable} without changing the object
  ({\defemph{$\alpha$-conversion}\index{alpha@$\alpha$-conversion}}). If {\snippet{<OMV
      name="x"/>}} is not bound, then the scope of the variable cannot be reliably
  defined; so equality with other occurrences of the variable {\snippet{<OMV name="x"/>}}
  becomes an ill-defined problem.  We therefore discourage the use of unbound variables in
  {\omdoc}; they are very simple to avoid by using symbols instead, introducing suitable
  theories if necessary (see {\mysecref{theories}}).

{\mathml} goes a different route: the {\element[ns-elt=m]{csymbol}} and {\element[ns-elt=m]{ci}}
elements have content that is {\pmathml}, which is used for the presentation of the
variable or symbol name.\footnote{Note that surprisingly, the empty {\cmathml} elements
  are treated more in the {\openmath} spirit.}  While this gives us a much better handle
on presentation of objects with variables than {\openmath} (where we are basically forced
to make due with the ASCII\footnote{In the current {\openmath} standard, variable names
  are restricted to alphanumeric characters starting with a letter. Note that unlike with
  symbols, we cannot associate presentation information with variables via style sheets,
  since these are not globally unique (see {\mysecref{pres-bound}} for a discussion of the
  {\omdoc} solution to this problem).}  representation of the variable name), the question
of scope and equality becomes much more difficult: Are two variables (semantically) the
same, even if they have different colors, sizes, or font families? Again, for symbols the
situation is simpler, since the {\attribute[ns-elt=m]{definitionURL}{csymbol}} attribute on the
{\element[ns-elt=m]{csymbol}} element establishes a global identity criterion (two symbols are
equal, iff they have the same {\attribute[ns-elt=m]{definitionURL}{csymbol}} value (as URI
strings; see~\cite{BerFie:uri98}).) The second edition of the {\mathml} standard adopts
the same solution for bound variables: it recommends to annotate the {\element[ns-elt=m]{bvar}}
elements that declare the bound variable with an {\attribute[ns-elt=m]{id}{bvar}} attribute and
use the {\attribute[ns-elt=m]{definitionURL}{ci}} attribute on the {\twintoo{bound}{occurrence}s}
of the {\element[ns-elt=m]{ci}} element to point to those. The following example is taken
from~\cite{KohDev:bvm03}, which has more details.

\begin{lstlisting}[language=MathML,label=bvar-mathml,
     index={math,bvar,ci,definitionURL}]
<m:lambda>
  <m:bvar><m:ci xml:id="the-boundvar">complex presentation</m:ci></m:bvar>
  <m:apply>
    <m:plus/>
    <m:ci definitionURL="#the-boundvar">complex presentation</m:ci>
    <m:ci definitionURL="#the-boundvar">complex presentation</m:ci>
  </m:apply>
</m:lambda>  
\end{lstlisting}

For presentation in {\mathml}, this gives us the best of both approaches, the
{\element[ns-elt=m]{ci}} content can be used, and the {\indextoo{pointer}} gives a simple
semantic equivalence criterion. For presenting {\openmath} and {\cmathml} in other
formats {\omdoc} makes use of the infrastructure introduced in module
{\PRESmodule{spec}}; see {\mysecref{pres-bound}} for a discussion.
\end{tsection}

\begin{tsection}[id=legacy]{Legacy Representation for Migration}

  Sometimes, {\omdoc} is used as a migration format from {\indextoo{legacy}} texts (see
  {\mychapref{algebra}} for an example). In such documents it can be too much effort to
  convert all mathematical objects and formulae into {\openmath} or {\cmathml} form. For
  this situation {\omdoc} provides the {\eldef{legacy}} element, which can contain
  arbitrary math markup\footnote{If the content is an {\xml}-based, format like Scalable
    Vector Graphics~\cite{W3C:svg02}, the {\indextoo{DTD}} must be augmented accordingly
    for validation.}. The {\element{legacy}} element can occur wherever an
  {\element[ns-elt=om]{OMOBJ}} or {\element[ns-elt=m]{math}} can and has an optional
  {\attribute[ns-attr=xml]{id}{legacy}} attribute for identification. The content is
  described by a pair of attributes:
\begin{itemize}
\item {\attribute{format}{legacy}} (required) specifies the format of the content
  using URI reference. {\omdoc} does not restrict the possible values, possible values
  include {\attval{TeX}{format}{legacy}}, {\attval{pmml}{format}{legacy}},
  {\attval{html}{format}{legacy}}, and {\attval{qmath}{format}{legacy}}.
\item {\attribute{formalism}{legacy}} is optional and describes the formalism (if
  applicable) the content is expressed in. Again, the value is unrestricted character data
  to allow a {\twintoo{URI}{reference}} to a definition of a formalism.
\end{itemize}

For instance in the following {\element{legacy}} element, the identity function is encoded
in the untyped $\lambda$-calculus, which is characterized by a reference to the relevant
Wikipedia article.

\begin{lstlisting}[index={legacy}]
<legacy format="TeX" formalism="http://en.wikipedia.org/wiki/Lambda_calculus">
  \lambda{x}{x}
</legacy>
\end{lstlisting}
\end{tsection}
\end{tchapter}
%%% Local Variables: 
%%% mode: latex
%%% TeX-master: "omdoc"
%%% End: 

% LocalWords:  pmml qmath mobjtable xref cd transc var quant arith geq OMB OMF
% LocalWords:  OMSTR OME po OMATP OMFOREIGN om OMR arcangle nodesep openmath emph
% LocalWords:  nat xlink href lst foo baz cmml ci cn csymbol definitionURL bvar
% LocalWords:  sts setname eq mathml mapsto truecm PCDATA dag sem alt ascii dec
% LocalWords:  pres omvscmml OMel dtd ref CMel BNF forall underdefined sec rep
% LocalWords:  omattr svg xmlns boundvar mobj cdbase omcds sinx allxsinx ome ns
% LocalWords:  omvsmom omvsmm elt mathescape attr nt OMOBJ OMV OMA OMBIND OMI
% LocalWords:  OMBVAR omcd CDDefinition leq nums nd th
