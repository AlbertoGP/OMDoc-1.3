%%%%%%%%%%%%%%%%%%%%%%%%%%%%%%%%%%%%%%%%%%%%%%%%%%%%%%%%%%%%%%%%%%%%%%%%%
% This file is part of the LaTeX sources of the OMDoc 1.3 specification
% Copyright (c) 2006 Michael Kohlhase
% This work is licensed by the Creative Commons Share-Alike license
% see http://creativecommons.org/licenses/by-sa/2.5/ for details
%%%%%%%%%%%%%%%%%%%%%%%%%%%%%%%%%%%%%%%%%%%%%%%%%%%%%%%%%%%%%%%%%%%%%%%%%

\begin{tsection}[id=changes1.3]{Changes from 1.2 to 1.3}

  The main change from {\omdocv{1.2}} to {\omdocv{1.3}} is the use of the new notation
  framework described in {\mychapref{pres}}. It completely replaces the presentation
  archicture of {\omdocv{1.2}}.

  The other large change is to use the new namespace \url{http://omdoc.org/ns} that will
  also be used in {\omdocv{1.2}}

\begin{center}\footnotesize
\begin{longtable}{|l|c|p{6cm}|l|}\hline
  element & state & comments & cf.\\\hline\hline 
{\element{dd}}           & cha
     & description items now allow block content as in XHTML
     & \mysecref{mtxt} \\\hline
{\element{bibliography}}           & new
     & generates the references
     & \mysecref{frontbackmatter} \\\hline
{\element{citation}}           & new
     & marks up a citation
     & \mysecref{mtxt} \\\hline
{\element{index}}           & new
     & generates the index
     & \mysecref{frontbackmatter} \\\hline
{\element{li}}           & cha
     & list items now allow block content as in XHTML
     & \mysecref{mtxt} \\\hline
{\element{metadata}}        & cha
     & the optional attribute {\oldattribute{inherits}{metadata}{1.2}} dropped, it was never
     sufficiently defined.
     & \mysecref{metadata}\\\hline
{\oldelement{presentation}{1.2}}           & del
     & replaced by the {\element{notation}} element. 
     & \mychapref{pres} \\\hline
{\oldelement{style}{1.2}}           & del
     & obsolete, since it was never used.
     & \\\hline
{\element{tableofcontents}}           & new
     & generates the tableofcontents
     & \mysecref{frontbackmatter} \\\hline
{\oldelement{tgroup}{1.2}}           & del
     & replaced by the {\element{omgroup}} element, it turns out that with RelaxNG we can
     do the necessary validation of theory content after all. 
     & \mychapref{statements} \\\hline
{\element{uses}}           & new
     & opens a CD catalog.
     & \mychapref{mtext} \\\hline
\end{longtable}
\end{center}
\end{tsection}

%%% Local Variables: 
%%% mode: stex
%%% TeX-master: "omdoc"
%%% End: 
