%%%%%%%%%%%%%%%%%%%%%%%%%%%%%%%%%%%%%%%%%%%%%%%%%%%%%%%%%%%%%%%%%%%%%%%%%
% This file is part of the LaTeX sources of the OMDoc 1.3 specification
% Copyright (c) 2006 Michael Kohlhase
% This work is licensed by the Creative Commons Share-Alike license
% see http://creativecommons.org/licenses/by-sa/2.5/ for details
%%%%%%%%%%%%%%%%%%%%%%%%%%%%%%%%%%%%%%%%%%%%%%%%%%%%%%%%%%%%%%%%%%%%%%%%%

\begin{tchapter}[id=proofs,short=Representing Proofs]{Representing Proofs (Module {\PFmodule{spec}})}

  Proofs form an essential part of mathematics and modern sciences.  Conceptually, a
  {\defin{proof}} is a representation of uncontroversial evidence for the truth of an
  {\indextoo{assertion}}.
  
  The question of what exactly constitutes a proof has been controversially discussed (see
  e.g.~\cite{BarCoh:ecm01}). The clearest (and most radical) definition is given by
  theoretical logic, where a proof is a sequence, or {\indextoo{tree}}, or
  {\atwintoo{directed}{acyclic}{graph}} ({\indextoo{DAG}})\atwin{directed}{acyclic}{graph}
  of applications of inference rules from a formally defined logical calculus, that meets
  a certain set of well-formedness conditions.  There is a whole zoo of logical
  calculi\twin{logical}{calculus} that are optimized for various applications. They have
  in common that they are extremely explicit and verbose, and that the proofs even for
  simple theorems can become very large. The advantage of having formal and fully explicit
  proofs is that they can be very easily verified, even by simple computer programs.  We
  will come back to this notion of {\indextoo{proof}} in {\mysecref{proofobjects}}.

  In mathematical practice the notion of a proof is more flexible, and more geared for
  consumption by humans: any line of argumentation is considered a proof, if it convinces
  its readers that it could in principle be expanded to a formal proof in the sense given
  above. As the expansion process is extremely tedious, this option is very seldom carried
  out explicitly. Moreover, as proofs are geared towards communication among humans, they
  are given at vastly differing levels of abstraction. From a very informal proof idea for
  the initiated specialist of the field, who can fill in the details herself, down to a
  very detailed account for skeptics or novices which will normally be still well above
  the formal level. Furthermore, proofs will usually be tailored to the specific
  characteristics of the audience, who may be specialists in one part of a proof while
  unfamiliar to the material in others. Typically such proofs have a
  sequence/tree/DAG-like structure, where the leaves are natural language sentences
  interspersed with mathematical formulae (or {\twintoo{mathematical}{vernacular}}).

  Let us consider a proof and its context (\myfigref{pf-example1-math}) as it could be
  found in a typical elementary math. textbook, only that we have numbered the proof steps
  for referencing convenience. {\Myfigref{pf-example1-math}} will be used as a running
  example throughout this chapter.

\begin{myfig}{pf-example1-math}{A Theorem with a Proof.}
\def\kasten{\hfil\null\nobreak\hfill
            \hbox{\vrule\vbox{\hrule width 6 pt\vskip 6pt\hrule}\vrule}
            \par\smallskip}
\fbox{\begin{minipage}{10cm}
    {\bf Theorem}: {\emph{There are infinitely many prime numbers.}}\\
    {\bf Proof}: We need to prove that the set $P$ of all prime numbers is not
    finite.
    \begin{center}
      \begin{tabular}{rp{8cm}}
        1. & We proceed by assuming that $P$ is finite and reaching a contradiction.\\
        2. & Let $P$ be finite.\\
        3. & Then $P=\{p_1,\ldots,p_n\}$ for some $p_i$.\\
        4. & Let $q \stackrel{def}{=} p_1 \cdots p_n + 1$.\\
        5. & Since for each $p_i \in P$ we have $q > p_i$, we conclude $q \notin P$.\\
        6. & We prove the absurdity by showing that $q$ is prime:\\
        7. & For each $p_i \in P$ we have $q = p_i k + 1$ for some
             natural number $k$, so $p_i$ can not divide $q$;\\
        8. & $q$ must be prime as $P$ is the set of all prime numbers. \\
        9. & Thus we have contradicted our assumption (2) \\
        10. & and proven the assertion.  \kasten
      \end{tabular}
    \end{center}
  \end{minipage}}
\end{myfig}

Since proofs can be marked up on several levels, we will introduce the {\omdoc}
markup for proofs in stages: We will first concentrate on proofs as structured
texts, marking up the discourse structure in example
{\myfigref{pf-example1-math}}. Then we will concentrate on the justifications of
proof steps, and finally we will discuss the scoping and hierarchical structure of
proofs.

The development of the representational infrastructure in {\omdoc} has a long history:
From the beginning the format strived to allow structural semantic markup for textbook
proofs as well as accommodate a wide range of formal proof systems without over-committing
to a particular system. However, the proof representation infrastructure from
{\vomdoc{1.1}} turned out not to be expressive enough to represent the proofs in the
{\sc{Helm}} library~\cite{AspPad:hsmw01}. As a consequence, the {\PFmodule{spec}} module
has been redesigned~\cite{AspKohSac:dtdop03} as part of the {\scsys{MoWGLI}}
project~\cite{AspKoht:mimp02}.  The current version of the {\PFmodule{spec}} module is an
adaptation of this proposal to be as compatible as possible with earlier versions of
{\omdoc}. It has been validated by interpreting it as an implementation of the
{\twintoo{$\overline\lambda\mu\tilde\mu$}{calculus}}~\cite{SacerdotiCoen:enlt05} proof
representation calculus.

\begin{tsection}[id=proof-text]{Proof Structure}
  In this section, we will concentrate on the structure of proofs apparent in the proof
  text and introduce the {\omdoc} infrastructure needed for marking up this aspect. Even
  if the proof in {\myfigref{pf-example1-math}} is very short and simple, we can observe
  several characteristics of a typical mathematical proof.  The proof starts with the
  thesis that is followed by nine main ``steps'' (numbered from 1 to 10). A very direct
  representation of the content of {\myfigref{pf-example1-math}} is given in
  {\mylstref{primes-omdoc-text}}.

\begin{lstlisting}[label=lst:primes-omdoc-text,mathescape,
  caption={An {\omdoc} Representation of {\myfigref{pf-example1-math}}.},
  index={symbol,definition}]
<assertion xml:id="a1">
  <CMP><xhtml:p>There are infinitely many prime numbers.</xhtml:p></CMP>
</assertion>
<proof xml:id="p" for="#a1">
  <omtext xml:id="intro">
    <CMP><xhtml:p>We need to prove that the set $P$ of all prime numbers is not finite.</xhtml:p></CMP>
  </omtext>
  <derive xml:id="d1">
    <CMP><xhtml:p>We proceed by assuming that $P$ is finite and reaching a contradiction.</xhtml:p></CMP>
    <method>
      <proof xml:id="p1">
        <hypothesis xml:id="h2"><CMP><xhtml:p>Let $P$ be finite.</xhtml:p></CMP></hypothesis>
        <derive xml:id="d3">
          <CMP><xhtml:p>Then $P=\{p_1,\ldots,p_n\}$ for some $p_i$.</xhtml:p></CMP>
          <method><premise xref="#h2"/></method>
        </derive>
        <symbol name="q"/>
        <definition xml:id="d4" for="q" type="informal">
          <CMP><xhtml:p>Let $q \stackrel{def}{=} p_1 \cdots p_n + 1$</xhtml:p></CMP>
        </definition>
        <derive xml:id="d5">
          <CMP><xhtml:p> Since for each $p_i\in P$ we have $q>p_i$, we conclude $q\notin P$.</xhtml:p></CMP>
        </derive>  
        <omtext xml:id="c6">
          <CMP><xhtml:p>We prove the absurdity by showing that $q$ is prime:</xhtml:p></CMP>
        </omtext>  
        <derive xml:id="d7">
          <CMP><xhtml:p>For each $p_i \in P$ we have $q = p_i k + 1$ for some
            natural number $k$, so $p_i$ can not divide $q$;</xhtml:p></CMP>
          <method><premise xref="#d4"/></method>
        </derive>
        <derive xml:id="d8">
          <CMP><xhtml:p>$q$ must be prime as $P$ is the set of all prime numbers.</xhtml:p></CMP> 
          <method><premise xref="#d7"/></method>
        </derive>
        <derive xml:id="d9">
          <CMP><xhtml:p>Thus we have contradicted our assumption</xhtml:p></CMP>
          <method><premise xref="#d5"/><premise xref="#d8"/></method>
        </derive>  
      </proof>
    </method>
  </derive>  
  <derive xml:id="d10" type="conclusion">
    <CMP><xhtml:p>This proves the assertion.</xhtml:p></CMP>
  </derive>  
</proof>
\end{lstlisting}

Proofs are specified by {\eldef{proof}} elements in {\omdoc} that have the optional
attributes {\attribute[ns-attr=xml]{id}{proof}} and {\attribute{theory}{proof}} and the
required attribute {\attribute{for}{proof}}. The {\attribute{for}{proof}} attribute points
to the assertion that is justified by this proof (this can be an {\element{assertion}}
element or a {\element{derive}} proof step (see below), thereby making it possible to
specify expansions of justifications and thus hierarchical proofs). Note that there can be
more than one proof for a given assertion.

\begin{myfig}{qtproof}{The {\omdoc} Proof Elements}
\begin{scriptsize}
\begin{tabular}{|>{\tt}l|>{\tt}l|>{\tt}p{2.6truecm}|c|>{\tt}p{4.2truecm}|}\hline
{\rm Element}& \multicolumn{2}{l|}{Attributes\hspace*{2.25cm}} & D & Content  \\\hline
             & {\rm Req.}  & {\rm Optional}     & C &           \\\hline\hline
 proof       & for         & theory, xml:id, class, style & +   
             & (omtext | derive | hypothesis | symbol | definition)* \\\hline
 proofobject & for             & xml:id, class, style, theory & +  & CMP*, ({\mobjabbr}) \\\hline
 hypothesis  &                 & xml:id, class, style, inductive & -- & CMP*, FMP*  \\\hline
 derive      &                 & xml:id, class, style, type & -- & CMP*, FMP*, method? \\\hline
 method      &                 & xref & -- & ({\mobjabbr} | premise | proof | proofobject)* \\\hline
 premise     & xref            & rank & -- & EMPTY\\\hline
\end{tabular}
\end{scriptsize}
\end{myfig}

The content of a proof consists of a sequence of proof steps, whose {\indextoo{DAG}}
structure is given by cross-referencing\index{cross-reference}. These proof steps are
specified in four kinds of {\omdoc} elements:
\begin{description}
\item[{\element{omtext}}] {\omdoc} allows this element to allow for intermediate text in
  proofs that does not have to have a logical correspondence to a proof step, but e.g.
  guides the reader through the proof. Examples for this are remarks by the proof author,
  e.g.  an explanation why some other proof method will not work. We can see another
  example in {\mylstref{primes-omdoc-text}} in lines 5-7, where the comment gives a
  preview over the course of the proof.
\item[{\element{derive}}] elements specify normal proof steps that derive a new claim from
  already known ones, from {\indextoo{assertion}s} or {\indextoo{axiom}s} in the current
  theory, or from the {\indextoo{assumption}s} of the assertion that is under
  consideration in the proof.  See for example lines $12ff$ in
  {\mylstref{primes-omdoc-text}} for examples of {\element{derive}} proof steps that only
  state the local assertion. We will consider the specification of justifications in
  detail in {\mysecref{proofs:justifications}} below. The {\eldef{derive}} element carries
  an optional {\attribute[ns-attr=xml]{id}{derive}} attribute for identification and an
  optional {\attribute{type}{derive}} to single out special cases of proofs steps.
  
  The value {\attval{conclusion}{type}{derive}} is reserved for the concluding step of a
  proof\footnote{As the argumentative structure of the proof is encoded in the
    justification structure to be detailed in {\mysecref{proofs:justifications}}, the
    concluding step of a proof need not be the last child of a proof element.}, i.e. the
  one that derives the assertion made in the corresponding theorem.  
  
  The value {\attval{gap}{type}{derive}} is used for proof steps that are not justified
  (yet): we call them {\twindef{gap}{steps}}. Note that the presence of gap steps allows
  {\omdoc} to specify {\twintoo{incomplete}{proof}s} as proofs with gap steps.
\item[{\element{hypothesis}}] elements allow to specify {\twintoo{local}{assumption}s}
  that allow the hypothetical reasoning discipline needed for instance to specify proof by
  contradiction, by case analysis, or simply to show that $A$ implies $B$, by assuming $A$
  and then deriving $B$ from this local hypothesis. The scope of an hypothesis extends to
  the end of the {\element{proof}} element containing it. In
  {\mylstref{primes-omdoc-text}} the classification of step 2 from
  {\myfigref{pf-example1-math}} as the {\eldef{hypothesis}} element {\snippet{h2}} forces
  us to embed it into a {\element{derive}} element with a {\element{proof}} grandchild,
  making a structure apparent that was hidden in the original.
  
  An important special case of hypothesis is the case of
  ``{\twintoo{inductive}{hypothesis}}'', this can be flagged by setting the value of the
  attribute {\attribute{inductive}{hypothesis}} to {\attval{yes}{inductive}{hypothesis}};
  the default value is {\attval{no}{inductive}{hypothesis}}.
  \setbox0=\hbox{\element{symbol}/\element{definition}}\item[\box0] These elements allow
  to introduce new local symbols that are local to the containing {\element{proof}}
  element.  Their meaning is just as described in {\mysecref{definitions}}, only that the
  role of the {\element{axiom}} element described there is taken by the
  {\element{hypothesis}} element. In {\mylstref{primes-omdoc-text}} step 4 in the proof
  is represented by a {\element{symbol}}/{\element{definition}} pair. Like in the
  {\element{hypothesis}} case, the scope of this symbol extends to the end of the
  {\element{proof}} element containing it.
\end{description}

These elements contain an informal (natural language) representation of the proof step in
a multilingual {\element{CMP}} group\twin{multilingual}{group} and possibly an
{\element{FMP}} element that gives a formal representation of the claim made by this proof
step. A {\element{derive}} element can furthermore contain a {\element{method}} element
that specifies how the assertion is derived from already-known facts (see the next section
for details). All of the proof step elements have an optional
{\attributeshort[ns-attr=xml]{id}} attribute for identification and the {\css} attributes.

As we have seen above, the content of any proof step is essentially a Gentzen-style sequent; see
{\mylstref{expansion}} for an example. This mixed representation enhances multi-modal
{\twintoo{proof}{presentation}}~\cite{Fiedler:tape97}, and the accumulation of proof
information in one structure. Informal proofs can be formalized~\cite{Baur:susmt99};
formal proofs can be transformed to natural language~\cite{HuangFiedler:pmfp96}. The first
is important, since it will be initially infeasible to totally formalize all
{\twintoo{mathematical}{proofs}} needed for the {\twintoo{correctness}{management}} of the
{\twintoo{knowledge}{base}}.
\end{tsection}

\begin{tsection}[id=proofs:justifications]{Proof Step Justifications}
  
  So far we have only concerned ourselves with the linear structure of the proof, we have
  identified the proof steps and classified them by their function in the proof. A central
  property of the {\element{derive}} elements is that their content (the local claim)
  follows from statements that we consider true. These can be earlier steps in the proof
  or general knowledge. To convince the reader of a proof, the steps are often accompanied
  with a {\defin{justification}}.  This can be given either by a logical
  {\twintoo{inference}{rule}} or {\twintoo{higher-level}{evidence}} for the truth of the
  claim.  The evidence can consist in a {\twintoo{proof}{method}} that can be used to
  prove the assertion, or in a separate subproof, that could be presented if the consumer
  was unconvinced.  Conceptually, both possibilities are equivalent, since the
  {\indextoo{method}} can be used to compute the subproof (called its
  {\defin{expansion}}). Justifications are represented in {\omdoc} by the
  {\element{method}} children of {\element{derive}} elements\footnote{The structural and
    formal justification elements discussed in this section are derived from hierarchical
    data structures developed for semi-automated theorem proving (satisfying the logical
    side). They allow natural language representations at every level (allowing for
    natural representation of mathematical vernacular at multiple levels of abstraction).
    This proof representation (see~\cite{BenzmuellerEtAl:otama97} for a discussion and
    pointers) is a DAG of nodes which represent the proof steps.}  (see
  {\mylstref{derive}} for an example):
  
  The {\eldef{method}} element contains a structural specification of the justification of
  the claim made in the {\element{FMP}} of a {\element{derive}} element. So the
  {\element{FMP}} together with the {\element{method}} element jointly form the
  counterpart to the natural language content of the {\element{CMP}} group, they are
  sibling to: The {\element{FMP}} formalizes the local claim, and the {\element{method}}
  stands for the justification. In {\mylstref{derive}} the formula in the {\element{CMP}}
  element corresponds to the claim, whereas the part ``By \ldots, we have'' is the
  justification. In other words, a {\element{method}} element specifies a proof method or
  inference rule with its arguments that justifies the assertion made in the
  {\element{FMP}} elements.  It has an optional {\attribute{xref}{method}} attribute whose
  target is an {\omdoc} definition of an {\twintoo{inference}{rule}} or
  {\twintoo{proof}{method}}.\footnote{At the moment {\omdoc} does not provide markup for
    such objects, so that they should best be represented by {\element{symbol}}s with
    {\element{definition}} where the inference rule is explained in the {\element{CMP}}
    (see the lower part of {\mylstref{derive}}), and the {\element{FMP}} holds a content
    representation for the inference rule, e.g.  using the content
    dictionary~\cite{CD:inference-rules}.  A good enhancement is to encapsulate
    system-specific encodings of the inference rules in {\element{private}} or
    {\element{code}} elements and have the {\attribute{xref}{method}} attribute point to
    these.} A method may have {\element[ns-elt=om]{OMOBJ}}, {\element[ns-elt=m]{math}},
  {\element{legacy}}, {\element{premise}}, {\element{proof}}, and
  {\element{proofobject}}\footnote{This object is an alternative representation of certain
    proofs, see {\mysecref{proofobjects}}.} children.  These act as
  {\indextoo{parameter}}s to the method, e.g. for the repeated universal instantiation
  method in {\mylstref{derive}} the parameters are the terms to instantiate the bound
  variables.
  
  The {\eldef{premise}} elements are used to refer to already established assertions:
  other proof steps or statements (given as {\element{assertion}}, {\element{definition}},
  or {\element{axiom}} elements) the method was applied to to obtain the local claim of
  the proof step. The {\element{premise}} elements are empty and carry the required
  attribute {\attribute{xref}{premise}}, which contains the URI of the assertion.  Thus
  the {\element{premise}} elements specify the {\indextoo{DAG}} structure of the
  proof. Note that even if we do not mark up the method in a justification (e.g. if it is
  unknown or obvious) it can still make sense to structure the argument in
  {\element{premise}} elements. We have done so in {\mylstref{primes-omdoc-text}} to make
  the dependencies of the argumentation explicit.
 
  If a {\element{derive}} step is a logically (or even mathematically) complex step, an
  expansion into sub-steps can be specified in a {\element{proof}} or
  {\element{proofobject}} element embedded into the justifying {\element{method}} element.
  An embedded proof allows us to specify generic markup for the hierarchic structure of
  proofs. Expansions of nodes justified by method applications are computed, but the
  information about the method itself is not discarded in the process as in tactical
  theorem provers like {\isabelle}~\cite{Paulson:iagtp94} or {\nuprl}~\cite{Constable86}.
  Thus, proof nodes may have justifications at multiple levels of abstraction in an
  hierarchical proof data structure.  Thus the {\element{method}} elements allow to
  augment the linear structure of the proof by a {\indextoo{tree}}/{\indextoo{DAG}}-like
  secondary structure given by the {\element{premise}} links. Due to the complex
  hierarchical structure of proofs, we cannot directly utilize the tree-like structure
  provided by {\xml}, but use cross-referencing\index{cross-reference}.  The
  {\element{derive}} step in {\mylstref{derive}} represents an inner node of the proof
  tree/DAG with three children (the elements with identifiers {\snippet{A2}},
  {\snippet{A4}}, and {\snippet{A5}}).

\begin{lstlisting}[label=lst:derive,mathescape,
  caption={A {\element{derive}} Proof Step},index={derive,method,premise}]
<proof xml:id="proof.2.1.2.proof.D2.1" for="#assertion.2.1.2">
  $\ldots$
  <derive xml:id="D2.1">
    <CMP><xhtml:p>By <oref xref="#A2"/>, <ref type="cite" xref="#A4"/>, and
       <oref xref="#A5"/> we have $z+(a+(-a))=(z+a)+(-a)$.</xhtml:p></CMP>
    <FMP>$z+(a+(-a))=(z+a)+(-a)$</FMP>
    <method xref="nk-sorts.omdoc#NK-Sorts.forallistar">
      <OMOBJ><OMV name="z"/></OMOBJ>
      <OMOBJ><OMV name="a"/></OMOBJ>
      <OMOBJ>$-a$</OMOBJ>
      <premise xref="#A2"/><premise xref="#A4"/><premise xref="#A5"/>
    </method>
  </derive>
  $\ldots$
</proof>
$\ldots$
<theory xml:id="NK-Sorts">
  <metadata>
    <dc:title>Natural Deduction for Sorted Logic</dc:title>
  </metadata>
  
  <symbol name="forallistar">
    <metadata>
      <dc:description>Repeated Universal Instantiation></dc:description>
    </metadata>
  </symbol>
  <definition xml:id="forallistar.def" for="forallistar" type="informal">
    <CMP><xhtml:p>Given $n$ parameters, the inference rule $\forall{I}^*$ instantiates 
      the first $n$ universal quantifications in the antecedent with them.</xhtml:p></CMP>
  </definition>
  $\ldots$
</theory>
\end{lstlisting}


In {\omdoc} the {\element{premise}} elements must reference proof steps in the current
proof or statements ({\element{assertion}} or {\element{axiom}} elements) in the scope of
the current theory: A statement is {\defemph{in scope of}}\index{scope}\twin{theory}{in
  scope of} the current theory, if its home theory is the current theory or imported
(directly or indirectly) by the current theory.
  
Furthermore note that a proof containing a {\element{premise}} element is not
self-contained evidence for the validity of the {\element{assertion}} it proves.
Of course it is only evidence for the validity at all (we call such a proof
{\indextoo{grounded}}), if all the statements that are targets of
{\element{premise}} references have grounded proofs themselves\footnote{For
  {\element{assertion}} targets this requirement is obvious. Obviously,
  {\element{axiom}s} do not need proofs, but certain forms of definitions need
  well-definedness proofs (see {\mysubsecref{definitions}}). These are included in
  the definition of a grounded proof.} and the reference relation does not contain
cycles. A grounded proof can be made self-contained by inserting the target
statements as {\element{derive}} elements before the referencing
{\element{premise}} and embedding at least one {\element{proof}} into the
{\element{derive}} as a justification.

Let us now consider another proof example ({\mylstref{expansion}}) to fortify our intuition.

\begin{lstlisting}[label=lst:expansion,mathescape,
  caption={An {\omdoc} Representation of a Proof by Cases},
  index={proof,derive,method,assumption,conclusion}]
<assertion xml:id="t1" theory="sets">
  <CMP><xhtml:p>If $a\in{U}$ or $a\in{V}$, then $a\in{U}\cup{V}$.</xhtml:p></CMP>
  <FMP>
    <assumption xml:id="t1_a">$a\in{U}\vee a\in{V}$</assumption>
    <conclusion xml:id="t1_c">$a\in{U}\cup{V}$</conclusion>
  </FMP>
</assertion>
<proof xml:id="t1_p1" for="#t1" theory="sets">
  <omtext xml:id="t1_p1_m1">
    <CMP><xhtml:p> We prove the assertion by a case analysis.</xhtml:p></CMP>
  </omtext>
  <derive xml:id="t1_p1_l1">
    <CMP><xhtml:p>If $a\in{U}$, then $a\in{U}\cup{V}$.</xhtml:p></CMP>
    <FMP>
      <assumption xml:id="t1_p1_l1_a">$a\in{U}$</assumption>
      <conclusion xml:id="t1_p1_l1_c">$a\in{U}\cup{V}$</conclusion>
    </FMP>
    <method xref="sk.omdoc#SK.by_definition">$\cup$</method>
  </derive> 
  <derive xml:id="t1_p1_l2">
    <CMP><xhtml:p>If $a\in{V}$, then $a\in{U}\cup{V}$.</xhtml:p></CMP>
    <FMP>
      <assumption xml:id="t1_p1_l2_a">$a\in{V}$</assumption>
      <conclusion xml:id="t1_p1_l2_c">$a\in{U}\cup{V}$</conclusion>
    </FMP>
    <method xref="sk.omdoc#SK.by_definition">$\cup$</method>
  </derive> 
  <derive xml:id="t1_p1_c">
    <CMP><xhtml:p> We have considered both cases, so we have $a\in{U}\cup{V}$.</xhtml:p></CMP>
  </derive> 
</proof>
\end{lstlisting}
This proof is in {{\twindef{sequent}{style}}}\twin{proof}{sequent}: The statement of all
local claims is in self-contained {\element{FMP}s} that mark up the statement in
{\element{assumption}}/{\element{conclusion}} form, which makes the logical dependencies
explicit. In this example we use inference rules from the calculus ``SK'',Gentzen's
sequent calculus for {\atwintoo{classical}{first-order}{logic}}~\cite{Gentzen:uudlsiii35},
which we assume to be formalized in a theory {\snippet{SK}}.  Note that local assumptions
from the {\element{FMP}} should not be referenced outside the {\element{derive}} step they
were made in. In effect, the {\element{derive}} element serves as a grouping device for
local assumptions.

Note that the same effect as embedding a {\element{proof}} element into a
{\element{derive}} step can be obtained by specifying the {\element{proof}} at top-level
and using the optional {\attribute{for}{proof}} attribute to refer to the identity of the
enclosing proof step (given by its optional {\attribute[ns-attr=xml]{id}{derive}}
attribute), we have done this in the proof in {\mylstref{expansion2}}, which expands the
{\element{derive}} step with identifier {\snippet{t1\_p1\_l1}} in {\mylstref{expansion}}.

\begin{lstlisting}[label=lst:expansion2,mathescape,
  caption={An External Expansion of Step {\snippet{t\_1\_p1\_l1}} in {\mylstref{expansion}}},
  index={proof,derive,method,assumption,conclusion}]
<definition xml:id="union.def" for="union">
  <OMOBJ>$\allcdot{P,Q,x}{x\in P\cup Q\Leftrightarrow x\in{P}\vee x\in{Q}}$</OMOBJ>
</definition>

<proof xml:id="t1_p1_l1.exp" for="#t1_p1_l1">
  <derive xml:id="t1_p1_l1.d1">
    <FMP>
      <assumption xml:id="t1_p1_l1.d1.a">$a\in{U}$</assumption>
      <conclusion xml:id="t1_p1_l1.d1.c">$a\in{U}$</conclusion>
    </FMP>
    <method xref="sk.omdoc#SK.axiom"/>
  </derive>
  <derive xml:id="t1_p1_l1.l1.d2">
    <FMP>
      <assumption xml:id="t1_p1_l1.d2.a">$a\in{U}$</assumption>
      <conclusion xml:id="t1_p1_l1.d2.c">$a\in{U}\vee a\in{V}$</conclusion>
    </FMP>
    <method xref="sk.omdoc#SK.orR"><premise xref="#t1_p1_l1.d1"/></method>
  </derive>
  <derive xml:id="t1_p1_l1.d3">
    <FMP>
      <assumption xml:id="t1_p1_l1.d3.a">$a\in{U}\vee a\in{V}$</assumption>
      <conclusion xml:id="t1_p1_l1.d3.c">$a\in{U}\cup{V}$</conclusion>
    </FMP>
    <method xref="sk.omdoc#SK.definition-rl">$U$, $V$, $a$
      <premise xref="#unif.def"/>
    </method>
  </derive>
  <derive xml:id="t1_p1_l1.d4">
    <FMP>
      <assumption xml:id="t1_p1_l1.d3.a">$a\in{U}$</assumption>
      <conclusion xml:id="t1_p1_l1.d3.c">$a\in{U}\cup{V}$</conclusion>
    </FMP>
    <method xref="sk.omdoc#SK.cut">
      <premise xref="#t1_p1_l1.d2"/>
      <premise xref="#t1_p1_l1.d3"/>
    </method>
  </derive>
</proof>          
\end{lstlisting}
\end{tsection}

\begin{tsection}[id=proofs:scoping]{Scoping and Context in a Proof}
  
  Unlike the {\atwintoo{sequent}{style}{proof}s} we discussed in the last section, many
  informal proofs use the
  {\defemph{\atwintoo{natural}{deduction}{style}}}\atwin{natural}{deduction}{proof}~\cite{Gentzen:uudlsiii35},
  which allows to reason from local assumptions. We have already seen such hypotheses as
  {\element{hypothesis}} elements in {\mylstref{primes-omdoc-text}}. The main new feature
  is that hypotheses can be introduced at some point in the proof, and are discharged
  later.  As a consequence, they can only be used in certain parts of the proof.  The
  hypothesis is inaccessible for inference outside the nearest ancestor {\element{proof}}
  element of the {\element{hypothesis}}.
  
  Let us now reconsider the proof in {\myfigref{pf-example1-math}}. Some of the steps (2,
  3, 4, 5, 7) leave the thesis unmodified; these are called {\twindef{forward}{reasoning}}
  or {\atwindef{bottom-up}{proof}{step}s}, since they are used to derive new knowledge
  from the available one with the aim of reaching the conclusion.  Some other steps (1, 6)
  are used to conclude the (current) thesis by opening new subproofs, each one
  characterized with a new local thesis.  These steps are called
  {\twindef{backward}{reasoning}} or {\atwindef{top-down}{proof}{step}s} steps, since they
  are used to reduce a complex problem (proving the thesis) to several simpler problems
  (the subproofs).  In our example, both backward reasoning steps open just one new
  subproof: Step 1 reduces the goal to proving that the finiteness of $P$ implies a
  contradiction; step 5 reduces the goal to proving that $q$ is prime.
  
  Step 2 is used to introduce a new hypothesis, whose scope extends from the point where
  it is introduced to the end of the current subproof, covering also all the steps
  inbetween and in particular all subproofs that are introduced in these. In our example
  the scope of the hypothesis that $P$ is finite (step 2 in {\myfigref{pf-example1-math}})
  are steps 3 -- 8. In an {\twintoo{inductive}{proof}}, for instance, the scope of the
  {\twintoo{inductive}{hypothesis}} covers only the proof of the
  {\twintoo{inductive}{step}} and not the proof of the base case (independently from the
  order adopted to present them to the user).
  
  Step 4 is similar, it introduces a new symbol $q$, which is a
  {\twintoo{local}{declaration}} that has scope over lines 4 -- 9.  The difference between
  a hypothesis and a local declaration is that the latter is used to introduce a variable
  as a new element in a given set or type, whereas the former, is used to locally state
  some property of the variables in scope. For example, {\emph{``let $n$ be a natural
      number''}} is a declaration, while {\emph{``suppose $n$ to be a multiple of 2''}} is
  a hypothesis.  The introduction of a new hypothesis or local declaration should always
  be justified by a proof step that discharges it. In our example the declaration $P$ is
  discharged in step 10. Note that in contrast to the representation in
  {\mylstref{primes-omdoc-text}} we have chosen to view step 6 in
  {\myfigref{pf-example1-math}} as a top-down proof step rather than a proof comment.
  
  To sum up, every proof step is characterized by a current thesis and a
  {\emin{context}}, which is the set of all the local declarations, hypotheses,
  and local definitions in scope. Furthermore, a step can either introduce a new
  hypothesis, definition, or declaration or can just be a forward or backward
  reasoning step.  It is a forward reasoning {\element{derive}} step if it leaves the current
  thesis as it is.  It is a backward reasoning {\element{derive}} step if it opens new
  subproofs, each one characterized by a new thesis and possibly a new context.

\begin{lstlisting}[label=lst:primes-omdoc,mathescape,
  caption={A top-down Representation of the Proof in {\myfigref{pf-example1-math}}.},
  index={symbol,definition}]
<assertion xml:id="a1">
  <CMP><xhtml:p>There are infinitely many prime numbers.</xhtml:p></CMP>
</assertion>
<proof for="#a1">
  <omtext xml:id="c0">
    <CMP><xhtml:p>We need to prove that the set $P$ of all prime numbers is not finite.</xhtml:p></CMP>
  </omtext>
  <derive xml:id="d1">
    <CMP><xhtml:p> We proceed by assuming that $P$ is finite and reaching a contradiction.</xhtml:p></CMP>
    <method xref="nk.omdoc#NK.by-contradiction">
      <proof>
        <hypothesis xml:id="h2"><CMP><xhtml:p>Let $P$ be finite.</xhtml:p></CMP></hypothesis>
        <derive xml:id="d3"><CMP><xhtml:p>Then $P=\{p_1,\ldots,p_n\}$ for some $n$</xhtml:p></CMP></derive>
        <symbol name="q"/>
        <definition xml:id="d4" for="q" type="informal">
          <CMP><xhtml:p>Let $q \stackrel{def}{=} p_1 \cdots p_n + 1$</xhtml:p></CMP>
        </definition>
        <derive xml:id="d5a">
          <CMP><xhtml:p>For each $p_i\in P$ we have $q > p_i$</xhtml:p></CMP>
          <method xref="#Trivial"><premise xref="#d4"/></method>
        </derive>
        <derive xml:id="d5b">
          <CMP><xhtml:p>$q \notin P$</xhtml:p></CMP>
          <method xref="#Trivial"><premise xref="#d5"/></method>
        </derive>
        <derive xml:id="d6">
          <CMP><xhtml:p>We show absurdity by showing that $q$ is prime</xhtml:p></CMP>
          <FMP>$\bot$</FMP>
          <method xref="#Contradiction">
            <premise xref="#d5b"/>
            <proof>
              <derive xml:id="d7a">
                <CMP><xhtml:p>
                  For each $p_i \in P$ we have $q = p_i k + 1$ for a given natural number $k$.
                </xhtml:p></CMP>
                <method xref="#By_Definition"><premise xref="#d1"/></method>
              </derive>
              <derive xml:id="d7b">
                <CMP><xhtml:p>Each $p_i \in P$ does not divide $q$</xhtml:p></CMP>
              </derive>
              <derive xml:id="d8">
                <CMP><xhtml:p>$q$ is prime</xhtml:p></CMP>
                <method xref="#Trivial">
                  <premise xref="#h2"/>
                  <premise xref="#p4"/>
                </method>
              </derive>
            </proof>
          </method>
        </derive>
      </proof>
    </method>
  </derive>
</proof>
\end{lstlisting}

{\element{proof}} elements are considered to be
non-assertional\twin{assertional}{element} in {\omdoc}, since they do not make
assertions about mathematical objects themselves, but only justify such assertions.
The assertional elements inside the proofs are governed by the scoping mechanisms
discussed there, so that using them in a context where assertional elements are
needed, can be forbidden. 
\end{tsection}

\begin{tsection}[id=proofobjects]{Formal Proofs as Mathematical Objects}
  
  In {\omdoc}, the notion of fully formal proofs is accommodated by the
  {\eldef{proofobject}} element. In logic, the term {\twindef{proof}{object}} is used for
  term representations for formal proofs via the Curry/Howard/DeBruijn Isomorphism (see
  e.g.~\cite{Thompson91} for an introduction and {\myfigref{proofobject}} for an example).
  $\lambda$-terms are among the most succinct representations of calculus-level proofs as
  they only document the inference rules. Since they are fully formal, they are very
  difficult to read and need specialized {\atwintoo{proof}{presentation}{system}s} for
  human consumption. In proof objects inference rules are represented as mathematical
  symbols, in our example in {\myfigref{proofobject}} we have assumed a theory
  {\snippet{PL0ND}} for the calculus of {\twintoo{natural}{deduction}} in
  {\twintoo{propositional}{logic}} which provides the necessary symbols (see
  {\mylstref{plnd}}).
  
  The {\element{proofobject}} element contains an optional multilingual group of
  {\element{CMP}} elements which describes the formal proof as well as a
  {\twintoo{proof}{object}} which can be an {\element[ns-elt=om]{OMOBJ}},
  {\element[ns-elt=m]{math}}, or {\element{legacy}} element.

\begin{myfig}{proofobject}{A Proof Object for the Commutativity of Conjunction}
\setbox0=\hbox{\quad\begin{textnd}
  \ian{\ibn{\ianc{[A\land B]}
                 {B}
                 {\land{E_r}\hspace{1em}}}
           {\ian{[A\land B]}
                {A}
                {\land{E_l}}}
           {B\land A}
           {\land{I}}}
       {A\land B\Rightarrow B\land A}
       {\Rightarrow\kern-.3em{I}}
\end{textnd}}
\setbox1=\hbox{\begin{minipage}{4.1cm}
\begin{lstlisting}[frame=none,numbers=none,mathescape,label=proofobject,
   index={proofobject,OMOBJ,OMBIND,OMS,OMBVAR,OMATTR,OMATP,OMA}]
<proofobject xml:id="ac.p" for="#and-comm">
 <metadata>
  <dc:description>
   Assuming $A\land B$ we have $B$ and $A$ 
   from which we can derive $B\land A$.
  </dc:description>
 </metadata>
 <OMOBJ>
  <OMBIND id="andcom.pf">
   <OMS cd="PL0ND" name="impliesI"/>
   <OMBVAR>
    <OMATTR>
     <OMATP>
      <OMS cd="PL0ND" name="type"/>
      $A\land{B}$
     </OMATP>
     <OMV name="X"/>
    </OMATTR>
   </OMBVAR>
   <OMA>
    <OMS cd="PL0ND" name="andI"/>
    <OMA>
     <OMA>
      <OMS cd="PL0ND" name="andEr"/>
      <OMV name="X"/>
     </OMA>
     <OMA>
      <OMS cd="PL0ND" name="andEl"/>
      <OMV name="X"/>
     </OMA>
    </OMA>
   </OMA>
  </OMBIND>
 </OMOBJ>
</proofobject>
\end{lstlisting}
\end{minipage}}
\setbox2=\hbox{\begin{minipage}{11cm}
  The schema on the left shows the proof as a natural deduction proof tree, the
  {\omdoc} representation gives the proof object as a $\lambda$ term. This term
  would be written as the following term in traditional (mathematical) notation:
  $\Rightarrow\kern-.3em{I}(\lambda{X:A\land{B}}.\land\kern-.3em{I}(\land{E_r}(X),\land{E_l}(X)))$
\end{minipage}}

\begin{tabular}{|cc|}\hline
 \box0 & \box1\\\hline
 \multicolumn{2}{|c|}{\box2}\\\hline
\end{tabular}
\end{myfig}
Note that using {\omdoc} symbols for inference rules and mathematical objects for proofs
reifies them to the object level and allows us to treat them at par with any other
mathematical objects. We might have the following theory for natural deduction in
propositional logic as a reference target for the second inference rule in
{\myfigref{proofobject}}.

\begin{lstlisting}[label=lst:plnd,mathescape,
  caption={A Theory for Propositional Natural Deduction}]
<theory xml:id="PL0ND">
  <metadata>
    <dc:description>The Natural Deduction Calculus for Propositional Logic</dc:description>
  </metadata>
  $\ldots$
  <symbol name="andI">
    <metadata><dc:subject>Conjunction Introduction</dc:subject></metadata>
    <type system="prop-as-types">$A\to B\to(A\land B)$</type>
  </symbol>

  <definition xml:id="andI.def" for="andi">
    <CMP><xhtml:p>Conjunction introduction, if we can derive $A$ and $B$, 
      then we can conclude $A\wedge B$.</xhtml:p></CMP>
  </definition>
  $\ldots$
</theory>
\end{lstlisting}

In particular, it is possible to use a {\element{definition}}
element to define a {\atwintoo{derived}{inference}{rule}} by simply specifying the proof
term as a {\indextoo{definiens}}:
\begin{lstlisting}[mathescape]
<symbol name="andcom">
  <metadata><dc:description>Commutativity for $\wedge$</dc:description></metadata>
  <type system="prop-as-types">$(A\land B)\to (B\land A)$</type>
</symbol>
<definition xml:id="andcom.def" for="#andcom" type="simple">
  <OMOBJ><OMR href="#andcom.pf"/></OMOBJ>
</definition>
\end{lstlisting}
Like {\element{proof}s}, {\element{proofobject}s} elements are considered to be
non-assertional\twin{assertional}{element} in {\omdoc}, since they do not make
assertions about mathematical objects themselves, but only justify such assertions.
\end{tsection}
\end{tchapter}

%%% Local Variables: 
%%% mode: latex
%%% TeX-master: "omdoc"
%%% End: 

% LocalWords:  pf lst mathescape metacomment qtproof xref foralli NK SK andi
% LocalWords:  orR pd proofobject em OMATP comm cd impliesI openproof andI def
% LocalWords:  andEr andEl proofobjects exp rl unif pt omdoc CMP FMP href andI
% LocalWords:  omtext truecm Gentzen OMOBJ OMV dc FOL OMBIND OMS OMBVAR OMATTR
% LocalWords:  OMA ac ref FoND andcom PL andi OMR Claudio's ns attr ff elt rp
% LocalWords:  MoWGLI forallistar strived Req multi metadata plnd andi andi
% LocalWords:  inbetween andi andI andI andI andI andI
