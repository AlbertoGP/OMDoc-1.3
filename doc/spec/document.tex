%%%%%%%%%%%%%%%%%%%%%%%%%%%%%%%%%%%%%%%%%%%%%%%%%%%%%%%%%%%%%%%%%%%%%%%%%
% This file is part of the LaTeX sources of the OMDoc 1.3 specification
% Copyright (c) 2016 Michael Kohlhase.
% Source at https://github.com/KWARC/OMDoc/tree/master/doc/spec
% This work is licensed by the Creative Commons Share-Alike license
% see http://creativecommons.org/licenses/by-sa/2.5/ for details
%%%%%%%%%%%%%%%%%%%%%%%%%%%%%%%%%%%%%%%%%%%%%%%%%%%%%%%%%%%%%%%%%%%%%%%%%

\begin{tchapter}[id=omdoc-infrastructure,short=Document Infrastructure]{Document Infrastructure (Module \DOCmodule{spec})}

Mathematical knowledge is largely communicated by way of a specialized set of
documents (e.g. e-mails, letters, pre-prints, journal articles, and textbooks).
These employ special notational conventions and visual
representations to convey the mathematical knowledge reliably and efficiently.

When marking up mathematical knowledge, one always has the choice whether to mark up the
structure of the document itself, or the structure of the mathematical knowledge that is
conveyed in the document. Even though in most documents, the document structure is
designed to help convey the structure of the knowledge, the two structures need not be the
same.  To frame the discussion we will distinguish two aspects of mathematical
documents. In the \twinemph{knowledge-centered}{view} we organize the mathematical
knowledge by its function, and do not care about a way to present it to human
recipients. In the \twinemph{narrative-centered}{view} we are interested in the
structure of the argument that is used to convey the mathematical knowledge to a human
user.

We will call a document \defin{knowledge-structured} and
\defemph{narrative-struc\-tu\-red}\index{narrative-structured}, based on which of the
two aspects is prevalent in the organization of the material.  Narrative-structured
documents in mathematics are generally directed at human consumption (even without being
in presentation markup). They have a general narrative structure: text interleaving with
formal elements like assertions, proofs, \ldots Generally, the order of presentation plays
a role in their effectiveness as a means of communication.  Typical examples of this class
are course materials or introductory textbooks.  Knowledge-structured documents are
generally directed at machine consumption or for referencing. They do not have a linear
narrative spine and can be accessed randomly and even re-ordered without information loss.
Typical examples of these are formula collections, {\openmath} content dictionaries,
technical specifications, etc.

The distinction between knowledge-structured and narrative-structured documents is
reminiscent of the \indextoo{presentation} vs.  \indextoo{content}
distinction\index{distinction!presentation vs. content} discussed in
\mysecref{math-objects}, but now it is on the level of document structure.  Note that
mathematical documents are often in both categories: a mathematical textbook can be read
from front to end, but it can also be used as a reference, accessing it by the index and
the table of contents.  The way humans work with knowledge also involves a change of
state. When we are taught or explore a mathematical domain, we have a linear/narrative
path through the material, from which we abstract more and more, finally settling for a
semantic representation that is relatively independent from the path we acquired it by.
Systems like \activemath} (see {\mysecref{activemath}) use the \omdoc format in
exactly that way playing on the difference between the two classes and generating
narrative-structured representations from knowledge-structured ones on the fly.

So, maybe the best way to think about this is that the question whether a document
is narrative- or knowledge-structured is not a property of the document itself,
but a property of the application processing this document.

\omdoc provides markup infrastructure for both aspects. In this chapter, we will discuss
the infrastructure for the narrative aspect --- for a working example we refer the reader
to \mychapref{courseware}. We will look at markup elements for knowledge-structured
documents in \mysecref{theories}.

Even though the infrastructure for narrative aspects of mathematical documents is somewhat
presentation-oriented, we will concentrate on content-markup for it. In particular, we
will not concern ourselves with questions like font families, sizes, alignment, or
positioning of text fragments. Like in most other {\xml} applications, this kind of
information can be specified in the \css} {\attributeshort{style} and
\attributeshort{class} attributes described in \mysecref{common-attribs}.

\begin{tsection}[id=root]{The Document Root}
  
  The {\xml} root element of the \omdoc format is the {\eldef{omdoc}} element, it
  contains all other elements described here. We call an \omdoc element a
  \twindef{top-level}{element}, if it can appear as a direct child of the
  \element{omdoc} element.
  
  The \element{omdoc} element (and the \element{omgroup} element introduced below as
  well) has an optional attribute \attribute[ns-attr=xml]{id}{omdoc} that can be used to
  reference the whole document. The \attribute{version}{omdoc} attribute is used to
  specify the version of the \omdoc format the file conforms to.  It is fixed to the
  string \snippet{1.3} by this specification.  This will prevent validation with a
  different version of the \indextoo{DTD} or \indextoo{schema}, or processing with an
  application using a different version of the \omdoc specification. The (optional)
  attribute \attributeshort{modules} allows to specify the \omdoc modules that are
  used in this document. The value of this attribute is a whitespace-separated list of
  module identifiers (e.g. \MOBJmodule{spec} the left column in
  \myfigref{omdoc-modules}), \omdoc sub-language identifiers (see
  \myfigref{omdoc-sub-languages}), or {\twintoo{URI}{reference}s} for externally given
  \omdoc modules or sub-language identifiers.\footnote{Allowing these external module
    references keeps the \omdoc format extensible. Like in the case with
    {\twintoo{namespace}{URI}s} \omdoc do not mandate that these
    {\twintoo{URI}{reference}s} reference an actual resource. They merely act as
    identifiers for the modules.} The intention is that if present, the
  \attributeshort{modules} specifies the list of all the modules used in the document
  (fragment).  If a \attributeshort{modules} attribute is present, then it is an error,
  if the content of this element contains elements from a module that is not specified;
  spurious module declarations in the \attribute{modules}{omdoc} attributes are allowed.
  
  The \element{omdoc} element acts as an implicit grouping element, just as the
  \element{omgroup} element to be introduced in \mysecref{sectioning}. Both have an
  optional \attribute{type}{omdoc} attribute; we will discuss its values and meaning in
  \mysecref{sectioning}.
  
  Here and in the following we will use tables as the one in \myfigref{qtgeneral} to
  give an overview over the respective \omdoc elements described in a chapter or
  section.  The first column gives the element name, the second and third columns specify
  the required and optional attributes. We will use the fourth column labeled ``DC'' to
  indicate whether an \omdoc element can have a {\element{metadata}} child, which will
  be described in the next section.  Finally the fifth column describes the content model
  --- i.e. the allowable children --- of the element. For this, we will use a form of
  Backus Naur Form notation\twin{Backus Naur form}{notation} also used in the DTD:
  \snippet{\#PCDATA} stands for ``\twintoo{parsed}{character data}'', i.e. text
  intermixed with legal \omdoc elements.) A synopsis of all elements is provided in
  \myappchapref{table}.

\begin{myfig}{qtgeneral}{\omdoc Elements for Specifying  Document Structure.}
\begin{scriptsize}
\begin{tabular}{|>{\tt}l|>{\tt}p{1truecm}|>{\tt}p{4truecm}|c|>{\tt}p{3.1truecm}|}\hline
{\rm Element}& \multicolumn{2}{l|}{Attributes\hspace*{2.25cm}} & D & Content  \\\hline
             & {\rm Required}  & {\rm Optional}      & C &           \\\hline\hline
 omdoc       &  version, xmlns 
                    & xml:id, type, class, style,  
                       version, modules, theory      & +  &
                       \llquote{front},(\llquote{top-level})*,\llquote{back} \\\hline
\llquote{back} & && index?,bibliography?\\\hline
\llquote{front} & && tableofcontents?\\\hline
 omgroup     &   & xml:id, modules, type, class, style, theory
                                                     & +  & (\llquote{top-level})* \\\hline
 metadata    &   & xml:id, class, style        & -- & \llquote{MDelt}*\\\hline
 ref         & xref & xml:id, type, class, style             & -- &     \\\hline
 ignore      &      & xml:id, type, comment                  & -- & ANY\\\hline
 index     &      & xml:id                  & -- & EMPTY\\\hline
 bibliography    & files  & xml:id                  & -- & EMPTY\\\hline
 \multicolumn{5}{|p{11cm}|}{where \llquote{top-level} stands for top-level \omdoc elements, and
   \llquote{MDelt} for those introduced in \mychapref{metadata}}\\\hline
\end{tabular}
\end{scriptsize}
\end{myfig}
\end{tsection}

\begin{newpart}{MK: we needed to introduce this, but the text is still very rough around
    the edges. Moreover, we should look at traditional front matter, which probably are
    things like abstract, acknowledgements, .... these need to be modeled as well. Also
    think about sectionalfrontmatter and sectional back matter, e.g. get inspired by the
    way LTXML treats them.}
\begin{tsection}[id=frontbackmatter]{Front/Backmatter}
  Documents usually have \twin{front}{matter} and \twin{back}{matter}, \omdoc is no
  exception. Currently, the \omdoc front matter only consists of the
  \element{tableofcontents} element. The back matter consists of the
  optional elements \element{index} and \element{bibliography}. 

  The \eldef{tableofcontents} element represents the position of an \twintoo{table
    of}{contents}\ednote{we should probably generalize this to a ``table of'' element
    where we can specify the ``of what'' in an attribute. } in the document. Note that
  since \omdoc is a source format, we do not actually have to put the contents of the
  table of contents at this position, but only need to specify content properties of the
  table of contents is intended; the actual content can be generated by the presentation
  process. For that the \element{tableofcontents} element uses the optional
  \attribute{level}{tableofcontents} that can be used to specify the depth of the table
  of contents.

 The \eldef{bibliography} element is similar to \element{index}, but it specifies the
  position bibliography to be generated. The \element{bibliography} element has a single
  required attribute: the \attribute{files}{bibliography} specifies the bibliography
  files in LaTeXML form\ednote{I think that we should just use the LaTeXML format, this
    works well, and we can create it. But we need to reference bibTeX and LaTeXML here}
  from which the actual references can be generated.\ednote{we also need the concept of a
    citation in the mtext module to make this work.}

  The \eldef{index} element represents the position of an index in the
  document.\ednote{MK: currently we have no way of specifying the content shape of the
    index, need to invent new attributes, maybe steal from LaTeXML}\ednote{maybe
    generalize for glossary, symbol index,...}
\end{tsection}
\end{newpart}

\begin{tsection}[id=metadata]{Metadata}
  
  The \twintoo{World Wide}{Web} was originally built for human consumption, and although everything
  on it is machine-readable, most of it is not machine-understandable.  The accepted
  solution is to provide \indextoo{metadata} (\twintoo{data}{about data}) to describe
  the documents on the web in a machine-understandable format that can be processed
  automatically. Metadata commonly specifies aspects of a document like title, authorship,
  language usage, and administrative aspects like modification dates, distribution rights,
  and identifiers.
  
  In general, metadata can either be embedded in the respective document, or be stated in
  a separate one. The first facilitates maintenance and control (metadata is always at
  your fingertips, and it can only be manipulated by the document's authors), the second
  one enables inference and distribution. \omdoc allows to embed metadata into the
  document, from where it can be harvested for external metadata formats, such as the
  {\xml} {\indextoo{resource description framework}}
  ({\rdf}~\cite{LasSwi:rdf99}).  We use one of the best-known metadata schemata for
  documents -- the \emin{Dublin Core} (cf.  \mysecsref{dc-elements}{dc-roles}). The
  purpose of annotating metadata in \omdoc is to facilitate the administration of
  documents, e.g.  digital \twintoo{rights}{management}, and to generate input for
  metadata-based tools, e.g.  RDF-based navigation and indexing of document collections.
  Unlike most other document formats \omdoc allows to add metadata at many levels, also
  making use of the metadata for document-internal markup purposes to ensure consistency.
  
  The \eldef{metadata} element contains elements for various metadata formats including
  bibliographic data from the Dublin Core vocabulary (as mentioned above), licensing
  information from the \twintoo{Creative Commons}{Initiative} (see
  \mysecref{creativecommons}), as well as information for {\openmath} content dictionary
  management. Application-specific metadata elements can be specified by adding
  corresponding \omdoc modules that extend the content model of the {\element{metadata}}
  element.

  The \omdoc {\element{metadata}} element can be used to provide information about the
  document as a whole (as the first child of the \element{omdoc} element), as well as
  about specific fragments of the document, and even about the top-level mathematical
  elements in \omdoc. This reinterpretation of bibliographic metadata as general data
  about knowledge items allows us to extract document fragments and re-assemble them to
  new aggregates without losing information about authorship, source, etc.
\end{tsection}

\begin{tsection}[id=comments]{Document Comments}
  
  Many content markup formats rely on {\indextoo{comment}ing} the source for human
  understanding; in fact {\twintoo{source}{comment}s} are considered a vital part of
  document markup. However, as {\xml} comments\twin{XML}{comment} (i.e. anything between
  ``\snippetin{<\char33--}'' and ``\snippetin{-->}'' in a document) need not even be
  read by some {\xml} parsers\twin{XML}{parser}, we cannot guarantee that they will
  survive any {\xml} manipulation of the \omdoc source.

  Therefore, anything that would normally go into comments should be modeled with an
  \element{omtext} element (\attribute{type}{omtext} \attval{comment}{type}{omtext},
  if it is a text-level comment; see \mysecref{omtext}) or with the \eldef{ignore}
  element for {\twintoo{persistent}{comment}s}, i.e.  comments that survive processing.
  The content of the \element{ignore} element can be any well-formed \omdoc, it can
  occur as an \omdoc top-level element or inside mathematical texts (see
  \mychapref{mtxt}). This element should be used if the author wants to comment the
  \omdoc representation, but the end user should not see their content in a final
  presentation of the document, so that \omdoc text elements are not suitable, e.g. in

\begin{lstlisting}[numbers=none,index={ignore},mathescape]
  <ignore type="todo" comment="this does not make sense yet, rework">
    <assertion xml:id="heureka">$\ldots$</assertion>
  </ignore>
\end{lstlisting}

Of course, \element{ignore} elements can be nested, e.g. if we want to mark up
the comment text (a pure string as used in the example above is not enough to
express the mathematics). This might lead to markup like 

\begin{lstlisting}[label=nested-ignore,numbers=none,index={ignore},mathescape]
  <ignore type="todo" comment="rework">
    <ignore type="todo-comment">
      <CMP><xhtml:p>This does not make sense yet, in particular, the equation 
        <OMOBJ>$\ldots$</OMOBJ> cannot be true, think of <OMOBJ>$\ldots$</OMOBJ>
      </xhtml:p></CMP>
    </ignore>
    <assertion xml:id="heureka">$\ldots$</assertion>
  </ignore>
\end{lstlisting}

Another good use of the \element{ignore} element is to use it as an analogon to the
\atwintoo{in-place}{error}{markup} in {\openmath} objects (see
\mysubsecref{om:error}). In this case, we use the \attribute{type}{ignore} attribute to
specify the kind of error and the content for the faulty \omdoc fragment. Note that
since the whole object must be a valid \omdoc object (or at least licensed by a DTD or
schema), the content itself must be a well-formed \omdoc fragment. As a consequence, the
\element{ignore} element can only be used for ``{\twintoo{mathematical}{error}s}'' like
sibling \element{CMP} or \element{FMP} elements that do not have the same meaning as
in \mylstref{ignore-error}. {\xml}-well-formedness and validity errors will have to be
handled by the {\xml} tools involved.

\begin{lstlisting}[label=lst:ignore-error,
  caption={Marking up Mathematical Errors Using \element{ignore}},
  numbers=none,index={ignore}]
  <ignore type="CMP-lang-error" 
          comment="multilingual CMPs are not translations of each other">
    <assertion xml:id="ass1">
      <CMP><xhtml:p>The proof is trivial.</xhtml:p></CMP>
      <CMP xml:lang="de"><xhtml:p>Der Beweis ist extrem schwer</xhtml:p></CMP>
    </assertion>
  </ignore>
\end{lstlisting}    
For another use of the \element{ignore} element, see \myfigref{flatten} in
\mysecref{sharing}.
\end{tsection}

\begin{tsection}[id=sectioning]{Document Structure}

  Like other documents mathematical ones are often divided into units like chapters,
  sections, and paragraphs by tags and nesting information. \omdoc makes these document
  relations explicit by using the \eldef{omgroup} element with an optional attribute
  \attribute{type}{omgroup}. It can take the values\footnote{\vomdoc{1.1} also allowed
    values \oldattval{dataset}{type}{omgroup}{1.2} and
    \oldattval{labeled-dataset}{type}{omgroup}{1.2} for marking up tables.  These values
    are deprecated in \vomdoc{1.2}, since we provide tables in module \RTmodule{spec};
    see \mysecref{rt} for details. Furthermore, \vomdoc{1.1} allowed the value
    \oldattval{narrative}{type}{omgroup}{1.2}, which was synonymous with
    \attval{sequence}{type}{omgroup}.}
\begin{description}
\item[\attval{sequence}{type}{omgroup}] for a succession of paragraphs. This is the
  default, and the normal way narrative texts are built up from paragraphs, mathematical
  statements, figures, etc. Thus, if no \attribute{type}{omgroup} is given the type
  \attval{sequence}{type}{omgroup} is assumed.
\item[\attval{itemize}{type}{omgroup}] for unordered lists. The children of this type of
  \element{omgroup} will usually be presented to the user as indented paragraphs preceded
  by a \twintoo{bullet}{symbol}. Since the choice of this symbol is purely presentational,
  \omdoc use the {\css} \attributeshort{style} or \attributeshort{class}
  attributes\twin{CSS}{attribute} on the children to specify the presentation of the
  bullet symbols (see \mysecref{common-attribs}).
\item[\attval{enumeration}{type}{omgroup}] for ordered lists. The children of this type of
  \element{omgroup} are usually presented like unordered lists, only that they are
  preceded by a running number of some kind (e.g. ``1.'', ``2.''\ldots or ``a)'',
  ``b)''\ldots; again the \attributeshort{style} or \attributeshort{class} attributes
  apply).
\item[\attval{sectioning}{type}{omgroup}] The children of this type of
  \element{omgroup} will be interpreted as sections. This means that the
  children will be usually numbered hierarchically, and their metadata will be
  interpreted as section heading information. For instance the
  \element{metadata}/\element[ns-elt=dc]{title} information (see \mysecref{dc-elements}
  for details) will be used as the section title. Note that \omdoc does not
  provide direct markup for particular hierarchical levels like
  ``\indextoo{chapter}'', ``\indextoo{section}'', or
  ``\indextoo{paragraph}'', but assumes that these are determined by the
  application that presents the content to the human or specified using the {\css}
  attributes.
\end{description}
Other values for the \attributeshort{type} attribute are also admissible, they should be
{\twintoo{URI}{reference}s} to documents explaining their intension.
  
We consider the \element{omdoc} element as an implicit \element{omgroup}, in order to
allow plugging together the content of different \omdoc documents as
{\element{omgroup}s} in a larger document. Therefore, all the attributes of the
\element{omdoc} element also appear on \element{omgroup} elements and behave exactly
like those.
\end{tsection}
\end{tchapter}

%%% Local Variables: 
%%% mode: latex
%%% TeX-master: "omdoc"
%%% End: 

% LocalWords:  xmlns RDF dtd lst ATTLIST DC qtmetadata ref ns attr specification itemize
% LocalWords:  otheromgrouptype xref mathescape enum qtgeneral dc activemath rt
% LocalWords:  ing openmath attribs xsi schemaLocation struc tu DCelement mtxt
% LocalWords:  todo heureka lang CMPs Der Beweis ist extrem schwer omdoc BNF cf
% LocalWords:  courseware namespace omgroup PCDATA truecm DOCTYPE CDATA xml CMP
% LocalWords:  omtext OMOBJ FMP de flattena flattenb kohlhase tchapter twinemph
% LocalWords:  MDelt creativecommons elt metadata Bachus Naur dataset DOCmodule
% LocalWords:  defin defemph ldots indextoo indextoo mysecref mychapref eldef specialized
% LocalWords:  attributeshort tsection twindef MOBJmodule myfigref twintoo tt
% LocalWords:  Backus Backus myappchapref myfig scriptsize tt tt tt hline rm organization
% LocalWords:  hspace llquote llquote llquote tableofcontents newpart ednote
% LocalWords:  sectionalfrontmatter frontbackmatter mtext LasSwi emin mysecsref
% LocalWords:  snippetin attval lstlisting atwintoo mysubsecref mylstref vomdoc
% LocalWords:  oldattval RTmodule hbox fbox leftrightarrow fbox SieBen acgap00
%  LocalWords:  narrative-centered well-formedness
