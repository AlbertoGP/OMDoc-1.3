%%%%%%%%%%%%%%%%%%%%%%%%%%%%%%%%%%%%%%%%%%%%%%%%%%%%%%%%%%%%%%%%%%%%%%%%%
% This file is part of the LaTeX sources of the OMDoc 1.3 specification
% Copyright (c) 2006 Michael Kohlhase
% This work is licensed by the Creative Commons Share-Alike license
% see http://creativecommons.org/licenses/by-sa/2.5/ for details
\svnInfo $Id: preface.tex 8685 2010-08-23 08:55:17Z kohlhase $
\svnKeyword $HeadURL: https://svn.omdoc.org/repos/omdoc/branches/omdoc-1.3/doc/spec/preface.tex $
%%%%%%%%%%%%%%%%%%%%%%%%%%%%%%%%%%%%%%%%%%%%%%%%%%%%%%%%%%%%%%%%%%%%%%%%%

\section*{Preface}\label{sec:intro}
\addcontentsline{toc}{section}{Preface} 

Mathematics is one of the oldest areas of human knowledge\footnote{We find mathematical
  knowledge written down on Sumerian clay tablets, and even
  {\indextoo{Euclid}}'s\twin{Euclid's}{elements} {\emph{Elements}}, an early rigorous
  development of a larger body of mathematics, is over 2000 years old.}.  It forms the
basis most modern sciences, technology and engineering disciplines build upon
it: Mathematics provides them with modeling tools like statistical analysis or
differential equations.  Inventions like public-key cryptography show that no part of
mathematics is fundamentally inapplicable.  Last, but not least, we teach mathematics to
our students to develop abstract thinking and hone their reasoning skills.
  
However, mathematical knowledge is far too vast to be understood by one person, moreover,
it has been estimated that the total amount of published mathematics doubles every
ten--fifteen years~\cite{Odlyzko:tlogr95}. Thus the question of supporting the management
and dissemination of mathematical
knowledge\atwin{mathematical}{knowledge}{management}\atwin{mathematical}{knowledge}{dissemination}
is becoming ever more pressing but remains difficult: Even though mathematical knowledge
can vary greatly in its presentation, level of formality and rigor, there is a level of
deep semantic structure that is common to all forms of mathematics and that must be
represented to capture the essence of the knowledge.

At the same time it is plausible to expect that the way we do (i.e. conceive, develop,
communicate about, and publish) mathematics will change considerably in the next years.
The Internet plays an ever-increasing role in our everyday life, and most of the
mathematical activities will be supported by mathematical software systems connected by a
commonly accepted distribution architecture, which makes the combined systems appear to the
user as one homogeneous application. They will communicate with human users and amongst
themselves by exchanging structured mathematical documents, whose document format makes
the context of the communication and the meaning of the mathematical objects unambiguous.

Thus the inter-operation of mathematical services can be seen as a knowledge management
task between software systems. On the other hand, mathematical knowledge management will
almost certainly be web-based, distributed, modular, and integrated into the emerging math
services architecture. So the two fields constrain and cross-fertilize each other at the
same time.  A shared fundamental task that has to be solved for the vision of a ``web of
mathematical knowledge'' ({\mathweb}) to become reality is to define an open markup
language for the mathematical objects and knowledge exchanged between mathematical
services.  The {\omdoc} format (\underline{O}pen \underline{M}athematical
\underline{Doc}uments) presented here is an answer to this challenge, it attempts to
provide an infrastructure for the communication and storage of
{\twintoo{mathematical}{knowledge}}.

Mathematics -- with its long tradition in the pursuit of {\twintoo{conceptual}{clarity}}
and {\twintoo{representational}{rigor}} -- is an interesting test case for general
{\twintoo{knowledge}{management}}, since it abstracts from vagueness of other knowledge
without limiting its inherent complexity. The concentration on mathematics in {\omdoc} and
this {\report} does not preclude applications in other areas. On the contrary, all the
material directly extends to the {\indextoo{STEM}} ({\indextoo{science}},
{\indextoo{technology}}, {\indextoo{education}}, and {\indextoo{mathematics}}) fields,
once a certain level of conceptualization has been reached.

This {\report} tries to be a one-stop information source about the {\omdoc} format, its
applications, and best practices. It is intended for authors of mathematical documents and
for application developers. The book is divided into four parts: an introduction to markup for
mathematics (Part I), an {\omdoc} primer with paradigmatic examples for many kinds of
mathematical documents (Part II), the rigorous specification of the {\omdoc} document
format (Part III), and an {\xml} document type definition and schema (Part IV).

The {\report} can be read in multiple ways: 
\begin{itemize}
\item for users that only need a casual exposure to the format, or authors that
  have a specific text category in mind, it may be best to look at the examples in
  the {\omdoc} primer ({\mypartref{primer}} of this {\report}), 
\item for an in-depth account of the format and all the possibilities of modeling
  mathematical documents, the rigorous specification in
  {\mypartref{specification}} is indispensable.  This is particularly true for
  application developers, who will also want to study the external resources,
  existing {\omdoc} applications and projects, in {\mypartref{applications}}.
\item Application developers will also need to familiarize themselves with the {\omdoc}
  Schema in the Appendix.
\end{itemize}

%%% Local Variables: 
%%% mode: stex
%%% TeX-master: "omdoc"
%%% End: 

% LocalWords:  athematical uments
